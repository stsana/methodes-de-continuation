\documentclass[11pt]{article}

    \usepackage[breakable]{tcolorbox}
    \usepackage{parskip} % Stop auto-indenting (to mimic markdown behaviour)
    
    \usepackage{iftex}
    \ifPDFTeX
    	\usepackage[T1]{fontenc}
    	\usepackage{mathpazo}
    \else
    	\usepackage{fontspec}
    \fi

    % Basic figure setup, for now with no caption control since it's done
    % automatically by Pandoc (which extracts ![](path) syntax from Markdown).
    \usepackage{graphicx}
    % Maintain compatibility with old templates. Remove in nbconvert 6.0
    \let\Oldincludegraphics\includegraphics
    % Ensure that by default, figures have no caption (until we provide a
    % proper Figure object with a Caption API and a way to capture that
    % in the conversion process - todo).
    \usepackage{caption}
    \DeclareCaptionFormat{nocaption}{}
    \captionsetup{format=nocaption,aboveskip=0pt,belowskip=0pt}

    \usepackage[Export]{adjustbox} % Used to constrain images to a maximum size
    \adjustboxset{max size={0.9\linewidth}{0.9\paperheight}}
    \usepackage{float}
    \floatplacement{figure}{H} % forces figures to be placed at the correct location
    \usepackage{xcolor} % Allow colors to be defined
    \usepackage{enumerate} % Needed for markdown enumerations to work
    \usepackage{geometry} % Used to adjust the document margins
    \usepackage{amsmath} % Equations
    \usepackage{amssymb} % Equations
    \usepackage{textcomp} % defines textquotesingle
    % Hack from http://tex.stackexchange.com/a/47451/13684:
    \AtBeginDocument{%
        \def\PYZsq{\textquotesingle}% Upright quotes in Pygmentized code
    }
    \usepackage{upquote} % Upright quotes for verbatim code
    \usepackage{eurosym} % defines \euro
    \usepackage[mathletters]{ucs} % Extended unicode (utf-8) support
    \usepackage{fancyvrb} % verbatim replacement that allows latex
    \usepackage{grffile} % extends the file name processing of package graphics 
                         % to support a larger range
    \makeatletter % fix for grffile with XeLaTeX
    \def\Gread@@xetex#1{%
      \IfFileExists{"\Gin@base".bb}%
      {\Gread@eps{\Gin@base.bb}}%
      {\Gread@@xetex@aux#1}%
    }
    \makeatother

    % The hyperref package gives us a pdf with properly built
    % internal navigation ('pdf bookmarks' for the table of contents,
    % internal cross-reference links, web links for URLs, etc.)
    \usepackage{hyperref}
    % The default LaTeX title has an obnoxious amount of whitespace. By default,
    % titling removes some of it. It also provides customization options.
    \usepackage{titling}
    \usepackage{longtable} % longtable support required by pandoc >1.10
    \usepackage{booktabs}  % table support for pandoc > 1.12.2
    \usepackage[inline]{enumitem} % IRkernel/repr support (it uses the enumerate* environment)
    \usepackage[normalem]{ulem} % ulem is needed to support strikethroughs (\sout)
                                % normalem makes italics be italics, not underlines
    \usepackage{mathrsfs}
    

    
    % Colors for the hyperref package
    \definecolor{urlcolor}{rgb}{0,.145,.698}
    \definecolor{linkcolor}{rgb}{.71,0.21,0.01}
    \definecolor{citecolor}{rgb}{.12,.54,.11}

    % ANSI colors
    \definecolor{ansi-black}{HTML}{3E424D}
    \definecolor{ansi-black-intense}{HTML}{282C36}
    \definecolor{ansi-red}{HTML}{E75C58}
    \definecolor{ansi-red-intense}{HTML}{B22B31}
    \definecolor{ansi-green}{HTML}{00A250}
    \definecolor{ansi-green-intense}{HTML}{007427}
    \definecolor{ansi-yellow}{HTML}{DDB62B}
    \definecolor{ansi-yellow-intense}{HTML}{B27D12}
    \definecolor{ansi-blue}{HTML}{208FFB}
    \definecolor{ansi-blue-intense}{HTML}{0065CA}
    \definecolor{ansi-magenta}{HTML}{D160C4}
    \definecolor{ansi-magenta-intense}{HTML}{A03196}
    \definecolor{ansi-cyan}{HTML}{60C6C8}
    \definecolor{ansi-cyan-intense}{HTML}{258F8F}
    \definecolor{ansi-white}{HTML}{C5C1B4}
    \definecolor{ansi-white-intense}{HTML}{A1A6B2}
    \definecolor{ansi-default-inverse-fg}{HTML}{FFFFFF}
    \definecolor{ansi-default-inverse-bg}{HTML}{000000}

    % commands and environments needed by pandoc snippets
    % extracted from the output of `pandoc -s`
    \providecommand{\tightlist}{%
      \setlength{\itemsep}{0pt}\setlength{\parskip}{0pt}}
    \DefineVerbatimEnvironment{Highlighting}{Verbatim}{commandchars=\\\{\}}
    % Add ',fontsize=\small' for more characters per line
    \newenvironment{Shaded}{}{}
    \newcommand{\KeywordTok}[1]{\textcolor[rgb]{0.00,0.44,0.13}{\textbf{{#1}}}}
    \newcommand{\DataTypeTok}[1]{\textcolor[rgb]{0.56,0.13,0.00}{{#1}}}
    \newcommand{\DecValTok}[1]{\textcolor[rgb]{0.25,0.63,0.44}{{#1}}}
    \newcommand{\BaseNTok}[1]{\textcolor[rgb]{0.25,0.63,0.44}{{#1}}}
    \newcommand{\FloatTok}[1]{\textcolor[rgb]{0.25,0.63,0.44}{{#1}}}
    \newcommand{\CharTok}[1]{\textcolor[rgb]{0.25,0.44,0.63}{{#1}}}
    \newcommand{\StringTok}[1]{\textcolor[rgb]{0.25,0.44,0.63}{{#1}}}
    \newcommand{\CommentTok}[1]{\textcolor[rgb]{0.38,0.63,0.69}{\textit{{#1}}}}
    \newcommand{\OtherTok}[1]{\textcolor[rgb]{0.00,0.44,0.13}{{#1}}}
    \newcommand{\AlertTok}[1]{\textcolor[rgb]{1.00,0.00,0.00}{\textbf{{#1}}}}
    \newcommand{\FunctionTok}[1]{\textcolor[rgb]{0.02,0.16,0.49}{{#1}}}
    \newcommand{\RegionMarkerTok}[1]{{#1}}
    \newcommand{\ErrorTok}[1]{\textcolor[rgb]{1.00,0.00,0.00}{\textbf{{#1}}}}
    \newcommand{\NormalTok}[1]{{#1}}
    
    % Additional commands for more recent versions of Pandoc
    \newcommand{\ConstantTok}[1]{\textcolor[rgb]{0.53,0.00,0.00}{{#1}}}
    \newcommand{\SpecialCharTok}[1]{\textcolor[rgb]{0.25,0.44,0.63}{{#1}}}
    \newcommand{\VerbatimStringTok}[1]{\textcolor[rgb]{0.25,0.44,0.63}{{#1}}}
    \newcommand{\SpecialStringTok}[1]{\textcolor[rgb]{0.73,0.40,0.53}{{#1}}}
    \newcommand{\ImportTok}[1]{{#1}}
    \newcommand{\DocumentationTok}[1]{\textcolor[rgb]{0.73,0.13,0.13}{\textit{{#1}}}}
    \newcommand{\AnnotationTok}[1]{\textcolor[rgb]{0.38,0.63,0.69}{\textbf{\textit{{#1}}}}}
    \newcommand{\CommentVarTok}[1]{\textcolor[rgb]{0.38,0.63,0.69}{\textbf{\textit{{#1}}}}}
    \newcommand{\VariableTok}[1]{\textcolor[rgb]{0.10,0.09,0.49}{{#1}}}
    \newcommand{\ControlFlowTok}[1]{\textcolor[rgb]{0.00,0.44,0.13}{\textbf{{#1}}}}
    \newcommand{\OperatorTok}[1]{\textcolor[rgb]{0.40,0.40,0.40}{{#1}}}
    \newcommand{\BuiltInTok}[1]{{#1}}
    \newcommand{\ExtensionTok}[1]{{#1}}
    \newcommand{\PreprocessorTok}[1]{\textcolor[rgb]{0.74,0.48,0.00}{{#1}}}
    \newcommand{\AttributeTok}[1]{\textcolor[rgb]{0.49,0.56,0.16}{{#1}}}
    \newcommand{\InformationTok}[1]{\textcolor[rgb]{0.38,0.63,0.69}{\textbf{\textit{{#1}}}}}
    \newcommand{\WarningTok}[1]{\textcolor[rgb]{0.38,0.63,0.69}{\textbf{\textit{{#1}}}}}
    
    
    % Define a nice break command that doesn't care if a line doesn't already
    % exist.
    \def\br{\hspace*{\fill} \\* }
    % Math Jax compatibility definitions
    \def\gt{>}
    \def\lt{<}
    \let\Oldtex\TeX
    \let\Oldlatex\LaTeX
    \renewcommand{\TeX}{\textrm{\Oldtex}}
    \renewcommand{\LaTeX}{\textrm{\Oldlatex}}
    % Document parameters
    % Document title
    \title{Projet}
    
    
    
    
    
% Pygments definitions
\makeatletter
\def\PY@reset{\let\PY@it=\relax \let\PY@bf=\relax%
    \let\PY@ul=\relax \let\PY@tc=\relax%
    \let\PY@bc=\relax \let\PY@ff=\relax}
\def\PY@tok#1{\csname PY@tok@#1\endcsname}
\def\PY@toks#1+{\ifx\relax#1\empty\else%
    \PY@tok{#1}\expandafter\PY@toks\fi}
\def\PY@do#1{\PY@bc{\PY@tc{\PY@ul{%
    \PY@it{\PY@bf{\PY@ff{#1}}}}}}}
\def\PY#1#2{\PY@reset\PY@toks#1+\relax+\PY@do{#2}}

\expandafter\def\csname PY@tok@w\endcsname{\def\PY@tc##1{\textcolor[rgb]{0.73,0.73,0.73}{##1}}}
\expandafter\def\csname PY@tok@c\endcsname{\let\PY@it=\textit\def\PY@tc##1{\textcolor[rgb]{0.25,0.50,0.50}{##1}}}
\expandafter\def\csname PY@tok@cp\endcsname{\def\PY@tc##1{\textcolor[rgb]{0.74,0.48,0.00}{##1}}}
\expandafter\def\csname PY@tok@k\endcsname{\let\PY@bf=\textbf\def\PY@tc##1{\textcolor[rgb]{0.00,0.50,0.00}{##1}}}
\expandafter\def\csname PY@tok@kp\endcsname{\def\PY@tc##1{\textcolor[rgb]{0.00,0.50,0.00}{##1}}}
\expandafter\def\csname PY@tok@kt\endcsname{\def\PY@tc##1{\textcolor[rgb]{0.69,0.00,0.25}{##1}}}
\expandafter\def\csname PY@tok@o\endcsname{\def\PY@tc##1{\textcolor[rgb]{0.40,0.40,0.40}{##1}}}
\expandafter\def\csname PY@tok@ow\endcsname{\let\PY@bf=\textbf\def\PY@tc##1{\textcolor[rgb]{0.67,0.13,1.00}{##1}}}
\expandafter\def\csname PY@tok@nb\endcsname{\def\PY@tc##1{\textcolor[rgb]{0.00,0.50,0.00}{##1}}}
\expandafter\def\csname PY@tok@nf\endcsname{\def\PY@tc##1{\textcolor[rgb]{0.00,0.00,1.00}{##1}}}
\expandafter\def\csname PY@tok@nc\endcsname{\let\PY@bf=\textbf\def\PY@tc##1{\textcolor[rgb]{0.00,0.00,1.00}{##1}}}
\expandafter\def\csname PY@tok@nn\endcsname{\let\PY@bf=\textbf\def\PY@tc##1{\textcolor[rgb]{0.00,0.00,1.00}{##1}}}
\expandafter\def\csname PY@tok@ne\endcsname{\let\PY@bf=\textbf\def\PY@tc##1{\textcolor[rgb]{0.82,0.25,0.23}{##1}}}
\expandafter\def\csname PY@tok@nv\endcsname{\def\PY@tc##1{\textcolor[rgb]{0.10,0.09,0.49}{##1}}}
\expandafter\def\csname PY@tok@no\endcsname{\def\PY@tc##1{\textcolor[rgb]{0.53,0.00,0.00}{##1}}}
\expandafter\def\csname PY@tok@nl\endcsname{\def\PY@tc##1{\textcolor[rgb]{0.63,0.63,0.00}{##1}}}
\expandafter\def\csname PY@tok@ni\endcsname{\let\PY@bf=\textbf\def\PY@tc##1{\textcolor[rgb]{0.60,0.60,0.60}{##1}}}
\expandafter\def\csname PY@tok@na\endcsname{\def\PY@tc##1{\textcolor[rgb]{0.49,0.56,0.16}{##1}}}
\expandafter\def\csname PY@tok@nt\endcsname{\let\PY@bf=\textbf\def\PY@tc##1{\textcolor[rgb]{0.00,0.50,0.00}{##1}}}
\expandafter\def\csname PY@tok@nd\endcsname{\def\PY@tc##1{\textcolor[rgb]{0.67,0.13,1.00}{##1}}}
\expandafter\def\csname PY@tok@s\endcsname{\def\PY@tc##1{\textcolor[rgb]{0.73,0.13,0.13}{##1}}}
\expandafter\def\csname PY@tok@sd\endcsname{\let\PY@it=\textit\def\PY@tc##1{\textcolor[rgb]{0.73,0.13,0.13}{##1}}}
\expandafter\def\csname PY@tok@si\endcsname{\let\PY@bf=\textbf\def\PY@tc##1{\textcolor[rgb]{0.73,0.40,0.53}{##1}}}
\expandafter\def\csname PY@tok@se\endcsname{\let\PY@bf=\textbf\def\PY@tc##1{\textcolor[rgb]{0.73,0.40,0.13}{##1}}}
\expandafter\def\csname PY@tok@sr\endcsname{\def\PY@tc##1{\textcolor[rgb]{0.73,0.40,0.53}{##1}}}
\expandafter\def\csname PY@tok@ss\endcsname{\def\PY@tc##1{\textcolor[rgb]{0.10,0.09,0.49}{##1}}}
\expandafter\def\csname PY@tok@sx\endcsname{\def\PY@tc##1{\textcolor[rgb]{0.00,0.50,0.00}{##1}}}
\expandafter\def\csname PY@tok@m\endcsname{\def\PY@tc##1{\textcolor[rgb]{0.40,0.40,0.40}{##1}}}
\expandafter\def\csname PY@tok@gh\endcsname{\let\PY@bf=\textbf\def\PY@tc##1{\textcolor[rgb]{0.00,0.00,0.50}{##1}}}
\expandafter\def\csname PY@tok@gu\endcsname{\let\PY@bf=\textbf\def\PY@tc##1{\textcolor[rgb]{0.50,0.00,0.50}{##1}}}
\expandafter\def\csname PY@tok@gd\endcsname{\def\PY@tc##1{\textcolor[rgb]{0.63,0.00,0.00}{##1}}}
\expandafter\def\csname PY@tok@gi\endcsname{\def\PY@tc##1{\textcolor[rgb]{0.00,0.63,0.00}{##1}}}
\expandafter\def\csname PY@tok@gr\endcsname{\def\PY@tc##1{\textcolor[rgb]{1.00,0.00,0.00}{##1}}}
\expandafter\def\csname PY@tok@ge\endcsname{\let\PY@it=\textit}
\expandafter\def\csname PY@tok@gs\endcsname{\let\PY@bf=\textbf}
\expandafter\def\csname PY@tok@gp\endcsname{\let\PY@bf=\textbf\def\PY@tc##1{\textcolor[rgb]{0.00,0.00,0.50}{##1}}}
\expandafter\def\csname PY@tok@go\endcsname{\def\PY@tc##1{\textcolor[rgb]{0.53,0.53,0.53}{##1}}}
\expandafter\def\csname PY@tok@gt\endcsname{\def\PY@tc##1{\textcolor[rgb]{0.00,0.27,0.87}{##1}}}
\expandafter\def\csname PY@tok@err\endcsname{\def\PY@bc##1{\setlength{\fboxsep}{0pt}\fcolorbox[rgb]{1.00,0.00,0.00}{1,1,1}{\strut ##1}}}
\expandafter\def\csname PY@tok@kc\endcsname{\let\PY@bf=\textbf\def\PY@tc##1{\textcolor[rgb]{0.00,0.50,0.00}{##1}}}
\expandafter\def\csname PY@tok@kd\endcsname{\let\PY@bf=\textbf\def\PY@tc##1{\textcolor[rgb]{0.00,0.50,0.00}{##1}}}
\expandafter\def\csname PY@tok@kn\endcsname{\let\PY@bf=\textbf\def\PY@tc##1{\textcolor[rgb]{0.00,0.50,0.00}{##1}}}
\expandafter\def\csname PY@tok@kr\endcsname{\let\PY@bf=\textbf\def\PY@tc##1{\textcolor[rgb]{0.00,0.50,0.00}{##1}}}
\expandafter\def\csname PY@tok@bp\endcsname{\def\PY@tc##1{\textcolor[rgb]{0.00,0.50,0.00}{##1}}}
\expandafter\def\csname PY@tok@fm\endcsname{\def\PY@tc##1{\textcolor[rgb]{0.00,0.00,1.00}{##1}}}
\expandafter\def\csname PY@tok@vc\endcsname{\def\PY@tc##1{\textcolor[rgb]{0.10,0.09,0.49}{##1}}}
\expandafter\def\csname PY@tok@vg\endcsname{\def\PY@tc##1{\textcolor[rgb]{0.10,0.09,0.49}{##1}}}
\expandafter\def\csname PY@tok@vi\endcsname{\def\PY@tc##1{\textcolor[rgb]{0.10,0.09,0.49}{##1}}}
\expandafter\def\csname PY@tok@vm\endcsname{\def\PY@tc##1{\textcolor[rgb]{0.10,0.09,0.49}{##1}}}
\expandafter\def\csname PY@tok@sa\endcsname{\def\PY@tc##1{\textcolor[rgb]{0.73,0.13,0.13}{##1}}}
\expandafter\def\csname PY@tok@sb\endcsname{\def\PY@tc##1{\textcolor[rgb]{0.73,0.13,0.13}{##1}}}
\expandafter\def\csname PY@tok@sc\endcsname{\def\PY@tc##1{\textcolor[rgb]{0.73,0.13,0.13}{##1}}}
\expandafter\def\csname PY@tok@dl\endcsname{\def\PY@tc##1{\textcolor[rgb]{0.73,0.13,0.13}{##1}}}
\expandafter\def\csname PY@tok@s2\endcsname{\def\PY@tc##1{\textcolor[rgb]{0.73,0.13,0.13}{##1}}}
\expandafter\def\csname PY@tok@sh\endcsname{\def\PY@tc##1{\textcolor[rgb]{0.73,0.13,0.13}{##1}}}
\expandafter\def\csname PY@tok@s1\endcsname{\def\PY@tc##1{\textcolor[rgb]{0.73,0.13,0.13}{##1}}}
\expandafter\def\csname PY@tok@mb\endcsname{\def\PY@tc##1{\textcolor[rgb]{0.40,0.40,0.40}{##1}}}
\expandafter\def\csname PY@tok@mf\endcsname{\def\PY@tc##1{\textcolor[rgb]{0.40,0.40,0.40}{##1}}}
\expandafter\def\csname PY@tok@mh\endcsname{\def\PY@tc##1{\textcolor[rgb]{0.40,0.40,0.40}{##1}}}
\expandafter\def\csname PY@tok@mi\endcsname{\def\PY@tc##1{\textcolor[rgb]{0.40,0.40,0.40}{##1}}}
\expandafter\def\csname PY@tok@il\endcsname{\def\PY@tc##1{\textcolor[rgb]{0.40,0.40,0.40}{##1}}}
\expandafter\def\csname PY@tok@mo\endcsname{\def\PY@tc##1{\textcolor[rgb]{0.40,0.40,0.40}{##1}}}
\expandafter\def\csname PY@tok@ch\endcsname{\let\PY@it=\textit\def\PY@tc##1{\textcolor[rgb]{0.25,0.50,0.50}{##1}}}
\expandafter\def\csname PY@tok@cm\endcsname{\let\PY@it=\textit\def\PY@tc##1{\textcolor[rgb]{0.25,0.50,0.50}{##1}}}
\expandafter\def\csname PY@tok@cpf\endcsname{\let\PY@it=\textit\def\PY@tc##1{\textcolor[rgb]{0.25,0.50,0.50}{##1}}}
\expandafter\def\csname PY@tok@c1\endcsname{\let\PY@it=\textit\def\PY@tc##1{\textcolor[rgb]{0.25,0.50,0.50}{##1}}}
\expandafter\def\csname PY@tok@cs\endcsname{\let\PY@it=\textit\def\PY@tc##1{\textcolor[rgb]{0.25,0.50,0.50}{##1}}}

\def\PYZbs{\char`\\}
\def\PYZus{\char`\_}
\def\PYZob{\char`\{}
\def\PYZcb{\char`\}}
\def\PYZca{\char`\^}
\def\PYZam{\char`\&}
\def\PYZlt{\char`\<}
\def\PYZgt{\char`\>}
\def\PYZsh{\char`\#}
\def\PYZpc{\char`\%}
\def\PYZdl{\char`\$}
\def\PYZhy{\char`\-}
\def\PYZsq{\char`\'}
\def\PYZdq{\char`\"}
\def\PYZti{\char`\~}
% for compatibility with earlier versions
\def\PYZat{@}
\def\PYZlb{[}
\def\PYZrb{]}
\makeatother


    % For linebreaks inside Verbatim environment from package fancyvrb. 
    \makeatletter
        \newbox\Wrappedcontinuationbox 
        \newbox\Wrappedvisiblespacebox 
        \newcommand*\Wrappedvisiblespace {\textcolor{red}{\textvisiblespace}} 
        \newcommand*\Wrappedcontinuationsymbol {\textcolor{red}{\llap{\tiny$\m@th\hookrightarrow$}}} 
        \newcommand*\Wrappedcontinuationindent {3ex } 
        \newcommand*\Wrappedafterbreak {\kern\Wrappedcontinuationindent\copy\Wrappedcontinuationbox} 
        % Take advantage of the already applied Pygments mark-up to insert 
        % potential linebreaks for TeX processing. 
        %        {, <, #, %, $, ' and ": go to next line. 
        %        _, }, ^, &, >, - and ~: stay at end of broken line. 
        % Use of \textquotesingle for straight quote. 
        \newcommand*\Wrappedbreaksatspecials {% 
            \def\PYGZus{\discretionary{\char`\_}{\Wrappedafterbreak}{\char`\_}}% 
            \def\PYGZob{\discretionary{}{\Wrappedafterbreak\char`\{}{\char`\{}}% 
            \def\PYGZcb{\discretionary{\char`\}}{\Wrappedafterbreak}{\char`\}}}% 
            \def\PYGZca{\discretionary{\char`\^}{\Wrappedafterbreak}{\char`\^}}% 
            \def\PYGZam{\discretionary{\char`\&}{\Wrappedafterbreak}{\char`\&}}% 
            \def\PYGZlt{\discretionary{}{\Wrappedafterbreak\char`\<}{\char`\<}}% 
            \def\PYGZgt{\discretionary{\char`\>}{\Wrappedafterbreak}{\char`\>}}% 
            \def\PYGZsh{\discretionary{}{\Wrappedafterbreak\char`\#}{\char`\#}}% 
            \def\PYGZpc{\discretionary{}{\Wrappedafterbreak\char`\%}{\char`\%}}% 
            \def\PYGZdl{\discretionary{}{\Wrappedafterbreak\char`\$}{\char`\$}}% 
            \def\PYGZhy{\discretionary{\char`\-}{\Wrappedafterbreak}{\char`\-}}% 
            \def\PYGZsq{\discretionary{}{\Wrappedafterbreak\textquotesingle}{\textquotesingle}}% 
            \def\PYGZdq{\discretionary{}{\Wrappedafterbreak\char`\"}{\char`\"}}% 
            \def\PYGZti{\discretionary{\char`\~}{\Wrappedafterbreak}{\char`\~}}% 
        } 
        % Some characters . , ; ? ! / are not pygmentized. 
        % This macro makes them "active" and they will insert potential linebreaks 
        \newcommand*\Wrappedbreaksatpunct {% 
            \lccode`\~`\.\lowercase{\def~}{\discretionary{\hbox{\char`\.}}{\Wrappedafterbreak}{\hbox{\char`\.}}}% 
            \lccode`\~`\,\lowercase{\def~}{\discretionary{\hbox{\char`\,}}{\Wrappedafterbreak}{\hbox{\char`\,}}}% 
            \lccode`\~`\;\lowercase{\def~}{\discretionary{\hbox{\char`\;}}{\Wrappedafterbreak}{\hbox{\char`\;}}}% 
            \lccode`\~`\:\lowercase{\def~}{\discretionary{\hbox{\char`\:}}{\Wrappedafterbreak}{\hbox{\char`\:}}}% 
            \lccode`\~`\?\lowercase{\def~}{\discretionary{\hbox{\char`\?}}{\Wrappedafterbreak}{\hbox{\char`\?}}}% 
            \lccode`\~`\!\lowercase{\def~}{\discretionary{\hbox{\char`\!}}{\Wrappedafterbreak}{\hbox{\char`\!}}}% 
            \lccode`\~`\/\lowercase{\def~}{\discretionary{\hbox{\char`\/}}{\Wrappedafterbreak}{\hbox{\char`\/}}}% 
            \catcode`\.\active
            \catcode`\,\active 
            \catcode`\;\active
            \catcode`\:\active
            \catcode`\?\active
            \catcode`\!\active
            \catcode`\/\active 
            \lccode`\~`\~ 	
        }
    \makeatother

    \let\OriginalVerbatim=\Verbatim
    \makeatletter
    \renewcommand{\Verbatim}[1][1]{%
        %\parskip\z@skip
        \sbox\Wrappedcontinuationbox {\Wrappedcontinuationsymbol}%
        \sbox\Wrappedvisiblespacebox {\FV@SetupFont\Wrappedvisiblespace}%
        \def\FancyVerbFormatLine ##1{\hsize\linewidth
            \vtop{\raggedright\hyphenpenalty\z@\exhyphenpenalty\z@
                \doublehyphendemerits\z@\finalhyphendemerits\z@
                \strut ##1\strut}%
        }%
        % If the linebreak is at a space, the latter will be displayed as visible
        % space at end of first line, and a continuation symbol starts next line.
        % Stretch/shrink are however usually zero for typewriter font.
        \def\FV@Space {%
            \nobreak\hskip\z@ plus\fontdimen3\font minus\fontdimen4\font
            \discretionary{\copy\Wrappedvisiblespacebox}{\Wrappedafterbreak}
            {\kern\fontdimen2\font}%
        }%
        
        % Allow breaks at special characters using \PYG... macros.
        \Wrappedbreaksatspecials
        % Breaks at punctuation characters . , ; ? ! and / need catcode=\active 	
        \OriginalVerbatim[#1,codes*=\Wrappedbreaksatpunct]%
    }
    \makeatother

    % Exact colors from NB
    \definecolor{incolor}{HTML}{303F9F}
    \definecolor{outcolor}{HTML}{D84315}
    \definecolor{cellborder}{HTML}{CFCFCF}
    \definecolor{cellbackground}{HTML}{F7F7F7}
    
    % prompt
    \makeatletter
    \newcommand{\boxspacing}{\kern\kvtcb@left@rule\kern\kvtcb@boxsep}
    \makeatother
    \newcommand{\prompt}[4]{
        \ttfamily\llap{{\color{#2}[#3]:\hspace{3pt}#4}}\vspace{-\baselineskip}
    }
    

    
    % Prevent overflowing lines due to hard-to-break entities
    \sloppy 
    % Setup hyperref package
    \hypersetup{
      breaklinks=true,  % so long urls are correctly broken across lines
      colorlinks=true,
      urlcolor=urlcolor,
      linkcolor=linkcolor,
      citecolor=citecolor,
      }
    % Slightly bigger margins than the latex defaults
    
    \geometry{verbose,tmargin=1in,bmargin=1in,lmargin=1in,rmargin=1in}
    
    

\begin{document}
    
    \maketitle
    
    

    
    

    \hypertarget{rapport-de-projet-danalyse-numuxe9riquegroupe-4}{%
\section{\texorpdfstring{{Rapport de Projet d'Analyse Numérique(Groupe
4)}}{Rapport de Projet d'Analyse Numérique(Groupe 4)}}\label{rapport-de-projet-danalyse-numuxe9riquegroupe-4}}

\hypertarget{thuxe8me-duxe9termination-du-ph-dun-acide-faible-et-muxe9thodes-de-continuation}{%
\subsection{\texorpdfstring{{Thème: Détermination du pH d'un acide
faible et méthodes de
continuation}}{Thème: Détermination du pH d'un acide faible et méthodes de continuation}}\label{thuxe8me-duxe9termination-du-ph-dun-acide-faible-et-muxe9thodes-de-continuation}}

\#\#\#

{Auteur: Sinclair TSANA (Etudiant en BAB2)\(\hspace{200px}\) Enseignant:
Mr.~Christophe TROESTLER}

    \hypertarget{objectifs-du-projet}{%
\subsection{\texorpdfstring{\textbf{Objectifs du
projet:}}{Objectifs du projet:}}\label{objectifs-du-projet}}

Dans ce projet, il est question d'aborder un problème concret grâce aux
techniques numériques vues au cours qui pourrons être adaptées de
manière à répondre à la série des 11 questions proposées.

\hypertarget{ruxe9sultats-attendues}{%
\subsection{\texorpdfstring{\textbf{Résultats
attendues:}}{Résultats attendues:}}\label{ruxe9sultats-attendues}}

\begin{quote}
\begin{enumerate}
\def\labelenumi{\arabic{enumi}.}
\tightlist
\item
  Un bref rapport écrit qui résume les développements mathématiques et
  les solutions apportées aux problèmes posées
\item
  Les conclusions des expériences numériques.
\end{enumerate}
\end{quote}

\textbf{Note:} Ce \href{https://jupyter.org/}{notebook} constitut une
réponse à ces deux points. Nous y présenterons à la fois le
\texttt{code\ python} ainsi que les détails de nos développements
mathématiques.

    \hypertarget{importation-des-librairies-nuxe9cessaires}{%
\subsection{\texorpdfstring{\textbf{Importation des librairies
nécessaires}}{Importation des librairies nécessaires}}\label{importation-des-librairies-nuxe9cessaires}}

    \begin{tcolorbox}[breakable, size=fbox, boxrule=1pt, pad at break*=1mm,colback=cellbackground, colframe=cellborder]
\prompt{In}{incolor}{1}{\boxspacing}
\begin{Verbatim}[commandchars=\\\{\}]
\PY{k+kn}{import} \PY{n+nn}{matplotlib}\PY{n+nn}{.}\PY{n+nn}{pyplot} \PY{k}{as} \PY{n+nn}{plt}   \PY{c+c1}{\PYZsh{} pour les representations graphiques}
\PY{k+kn}{import} \PY{n+nn}{math}                       \PY{c+c1}{\PYZsh{} contient les fonctions mathematiques usuels tels que log, sin, etc...}
\PY{k+kn}{import} \PY{n+nn}{random}
\PY{k+kn}{import} \PY{n+nn}{array}
\PY{k+kn}{import} \PY{n+nn}{numpy} \PY{k}{as} \PY{n+nn}{np}                \PY{c+c1}{\PYZsh{} pour les calculs numeriques}
\PY{k+kn}{from} \PY{n+nn}{sympy} \PY{k+kn}{import} \PY{o}{*}               \PY{c+c1}{\PYZsh{} pour les calculs symboliques}
\PY{k+kn}{from} \PY{n+nn}{math} \PY{k+kn}{import} \PY{n}{log10}
\PY{k+kn}{from} \PY{n+nn}{scipy}\PY{n+nn}{.}\PY{n+nn}{optimize} \PY{k+kn}{import} \PY{n}{brentq} 
\PY{k+kn}{from} \PY{n+nn}{scipy}\PY{n+nn}{.}\PY{n+nn}{integrate} \PY{k+kn}{import} \PY{n}{solve\PYZus{}ivp}
\PY{k+kn}{from} \PY{n+nn}{numpy}\PY{n+nn}{.}\PY{n+nn}{linalg} \PY{k+kn}{import} \PY{n}{inv}
\end{Verbatim}
\end{tcolorbox}

    On s'intéresse à la variation du pH d'une solution aqueuse d'un acide
faible AH (formé d'un atome d'hydrogène H et d'une autre partie notée A)
en fonction de \(−log_{10}c\) où \(c >0\) représente la concentration de
l'acide. L'analyse de ce problème conduit aux équations suivantes :

\[
\left\{
    \begin{array}{ll}
        [H_3O^+][OH^-]=k_e \hspace{100px}    (1) \\
        [H_3O^+]=[OH^-]+[A^-]  \hspace{65px}(2) \\
        [AH]k_a=[A^-][H_3O^+] \hspace{80px} (3) \\
        [AH]+[A^-]=c        \hspace{125px}   (4) 
    \end{array}
\right.
\] \#\# \textbf{Question 1} \#\#\# \textbf{1.a) Détermination du
polynôme}

    On procède par élimination successive de \([OH^-], [A^-]\) et \([AH]\)
des équations ci-dessus.

\((1)\Rightarrow[OH^-]=\frac{k_e}{[H_3O^+]}\)

Dans (2) on a:
\(\hspace{40px}[H_3O^+]=\frac{k_e}{[H_3O^+]}+[A^-] \iff[A^-]=[H_3O^+]-\frac{k_e}{[H_3O^+]} \hspace{40px}(*)\)

Dans (3) on a: \(\hspace{40px}[AH]=\frac{1}{k_a}([H_3O^+]^2-k_e\)

Dans (4) on a:
\(\hspace{40px}[A^-]=c-\frac{1}{k_a}([H_3O^+]^2-k_e)\hspace{40px}(**)\)

\((*)\) et \((**)\) donnent:
\(\hspace{40px}[H_3O^+]-\frac{k_e}{[H_3O^+]}=c-\frac{1}{k_a}([H_3O^+]^2-k_e)\)

En multipliant par \(k_a[H_3O^+]\), on obtient:
\(\hspace{40px}k_a[H_3O^+]^2-k_ak_e = ck_a[H_3O^+]-[H_3O^+]^3+k_e[H_3O^+]\)

\(\hspace{250px}\iff [H_3O^+]^3+k_a[H_3O^+]-(ck_a[H_3O^+]+k_e[H_3O^+])-k_ak_e=0\)

D'ou l'équation \(X^3+k_aX^2-(k_e+ck_a)X-k_ak_e=0\) Où X représente la
concentration en \(H_3O^+\)

    \hypertarget{montrons-que-le-polynuxf4me-p-possuxe8de-une-unique-racine-positive.}{%
\subsubsection{\texorpdfstring{\textbf{Montrons que le polynôme P
possède une unique racine
positive.}}{Montrons que le polynôme P possède une unique racine positive.}}\label{montrons-que-le-polynuxf4me-p-possuxe8de-une-unique-racine-positive.}}

On a:\(\hspace{40px}P(x) = x^3+k_ax^2-(k_e+ck_a)x-k_ak_e\)

Trouver les racines de P c'est résoudre l'équation
\(x^3+k_ax^2-(k_e+ck_a)x-k_ak_e=0\)

\(P(x)=0 \iff x=(-k_ax^2+(ck_a+k_e)x+k_ak_e)^{1/3}\)

Posons \(\hspace{40px}f(x)=-k_ax^2+(ck_a+k_e)x+k_ak_e\)

\begin{itemize}
\item
  \textbf{Etudions les variations de la fonction \(f\)}

  On a: \(\hspace{40px}f^{'}(x)=-2k_ax+ck_a+k_e\)

  \$f\^{}\{'\}(x)=0
  \iff x=x\_1=\frac{ck_a+k_e}{k_a}\textgreater{}\frac{k_e}{k_a}\textgreater{}0\hspace{40px},
  \hspace{40px}f\^{}\{"\}(x)=-2k\_a\textless{}0 \$ pour tout x réel
  positif.

  Ainsi \(\forall x\geq 0\), on a \(f^{'}(x)\leq ck_a+k_e\hspace{40px}\)
  et \(\hspace{40px}\forall x\in [0,x_1]\), on a
  \(k_ak_e\leq f(x)\leq f(x_1)\)

  Posons \(F(x)=(f(x))^{1/3}\), on a
  \(F^{'}(x)=\frac{1}{3}f^{'}(x)(f(x))^{-2/3}\leq \frac{1}{3}(ck_a+k_e)(f(x))^{-2/3}\)
\end{itemize}

Or, f est à valeurs strictement positives sur \([0,x_1]\) et la fonction
\(x:\longmapsto x^{-2/3}\) étant décroissante sur \(]0,+\infty[\), donc
la fonction \(g:x\longmapsto (f(x))^{-2/3}\) est bornée su \([0,x_1]\).
Pour tout \(0\leq x \leq x_1\), on a \(g(0)\geq g(x) \geq g(x_1)\). Donc
g est majorée sur \([0,x_1]\) par: \(g(0)=(f(0))^{-2/3}\) et on a
\(f(0)=k_ak_e\).

Ce qui nous donne \(F^{'}(x)\leq K\) où
\(K=\frac{1}{3}(ck_a+k_e)(k_ak_e)^{-2/3}\).

Ainsi, d'après le
\href{https://fr.wikipedia.org/wiki/Th\%C3\%A9or\%C3\%A8me_des_accroissements_finis?oldformat=true}{\textbf{Théorème
des Accroissements Finis}}, on a :

\(\forall x,y \in [0,x_1] \hspace{40px} \lvert F(x)-F(y) \rvert \leq K\lvert x-y \rvert\)

\(K\leq 1 \iff \frac{1}{3}.(ck_a+k_e)(k_ak_e)^{-2/3}\leq 1 \iff ck_a+k_e \leq 3(k_ak_e)^{2/3}\iff c\leq \frac{3}{k_a}.((k_ak_e)^{2/3}-k_e)\)

Posons \(c_1 = \frac{3}{k_a}.((k_ak_e)^{2/3}-k_e)\)

Donc la fonction F est K-Lipschitzienne sur \([0,x_1]\) et en
particulier elle est contractant sur \([0,x_1]\) lorsque
\(c\in [0, c_1]\)

Il en découle que F admet un unique point fixe. Autrement dit l'équation
\(F(x)=x\), qui est équivalente à \(P(x)=0\), admet une unique solution
sur \([0,x_1]\)

    \begin{tcolorbox}[breakable, size=fbox, boxrule=1pt, pad at break*=1mm,colback=cellbackground, colframe=cellborder]
\prompt{In}{incolor}{2}{\boxspacing}
\begin{Verbatim}[commandchars=\\\{\}]
\PY{c+c1}{\PYZsh{} Quelques valeurs particulières}
\PY{n}{k\PYZus{}a} \PY{o}{=} \PY{l+m+mf}{1.8}\PY{o}{*}\PY{l+m+mi}{10}\PY{o}{*}\PY{o}{*}\PY{p}{(}\PY{o}{\PYZhy{}}\PY{l+m+mi}{5}\PY{p}{)}     \PY{c+c1}{\PYZsh{} Constante d’acidité  propre à l’acide}
\PY{n}{k\PYZus{}e} \PY{o}{=} \PY{l+m+mf}{1.0116}\PY{o}{*}\PY{l+m+mi}{10}\PY{o}{*}\PY{o}{*}\PY{p}{(}\PY{o}{\PYZhy{}}\PY{l+m+mi}{14}\PY{p}{)} \PY{c+c1}{\PYZsh{} La constante ionique de l’eau}
\PY{n}{c1} \PY{o}{=} \PY{p}{(}\PY{l+m+mi}{3}\PY{o}{/}\PY{n}{k\PYZus{}a}\PY{p}{)}\PY{o}{*}\PY{p}{(}\PY{p}{(}\PY{n}{k\PYZus{}a}\PY{o}{*}\PY{n}{k\PYZus{}e}\PY{p}{)}\PY{o}{*}\PY{o}{*}\PY{p}{(}\PY{l+m+mi}{2}\PY{o}{/}\PY{l+m+mi}{3}\PY{p}{)}\PY{o}{\PYZhy{}}\PY{n}{k\PYZus{}e}\PY{p}{)} \PY{c+c1}{\PYZsh{} tels que posée ci\PYZhy{}dessus}
\PY{n}{x1\PYZus{}c1} \PY{o}{=} \PY{n}{c1} \PY{o}{+} \PY{n}{k\PYZus{}e}\PY{o}{/}\PY{n}{k\PYZus{}a}    \PY{c+c1}{\PYZsh{} valeur de x1 pour c\PYZlt{}=c1}
\PY{n}{c1}\PY{p}{,} \PY{n}{x1\PYZus{}c1}
\end{Verbatim}
\end{tcolorbox}

            \begin{tcolorbox}[breakable, size=fbox, boxrule=.5pt, pad at break*=1mm, opacityfill=0]
\prompt{Out}{outcolor}{2}{\boxspacing}
\begin{Verbatim}[commandchars=\\\{\}]
(5.1857032777617826e-08, 5.241903277761783e-08)
\end{Verbatim}
\end{tcolorbox}
        
    \hypertarget{b-algorithme-de-horner-pour-uxe9valuer-p}{%
\subsubsection{\texorpdfstring{\textbf{1.b) Algorithme de Horner pour
évaluer
P}}{1.b) Algorithme de Horner pour évaluer P}}\label{b-algorithme-de-horner-pour-uxe9valuer-p}}

    On considère le polynôme \(P(x)=a_nx^n+a_{n-1}x^{n-1}+...+a_1x^1+a_0\)

\textbf{La
\href{https://en.wikipedia.org/wiki/Horner\%27s_method?oldformat=true}{méthode
de Horner} consiste a écrire P(x) sous la forme \(P_1(x).x+a_0\) et de
repéter cette opération sur \(P_1\) et ainsi de suite jusqu'a obtenir le
polynôme constant \(P_n(x)=a_n\)}. Son implémentation en python est la
suivante:

    \begin{tcolorbox}[breakable, size=fbox, boxrule=1pt, pad at break*=1mm,colback=cellbackground, colframe=cellborder]
\prompt{In}{incolor}{3}{\boxspacing}
\begin{Verbatim}[commandchars=\\\{\}]
\PY{k}{def} \PY{n+nf}{horner}\PY{p}{(}\PY{n}{A}\PY{p}{,}\PY{n}{x}\PY{p}{)}\PY{p}{:} \PY{c+c1}{\PYZsh{} A est le vecteur contenant les coefficients (a\PYZus{}i)\PYZus{}\PYZob{}i=0...n\PYZcb{}}
    \PY{n}{n} \PY{o}{=} \PY{n+nb}{len}\PY{p}{(}\PY{n}{A}\PY{p}{)}\PY{o}{\PYZhy{}}\PY{l+m+mi}{1}
    \PY{n}{p} \PY{o}{=} \PY{n}{A}\PY{p}{[}\PY{n}{n}\PY{p}{]}
    \PY{n}{i} \PY{o}{=} \PY{n}{n}
    \PY{k}{for} \PY{n}{i} \PY{o+ow}{in} \PY{n+nb}{range}\PY{p}{(}\PY{n}{n}\PY{o}{\PYZhy{}}\PY{l+m+mi}{1}\PY{p}{,}\PY{l+m+mi}{0}\PY{p}{,}\PY{o}{\PYZhy{}}\PY{l+m+mi}{1}\PY{p}{)}\PY{p}{:}
        \PY{n}{p} \PY{o}{=} \PY{n}{p}\PY{o}{*}\PY{n}{x} \PY{o}{+} \PY{n}{A}\PY{p}{[}\PY{n}{i}\PY{p}{]}
    \PY{k}{return} \PY{n}{p}
\end{Verbatim}
\end{tcolorbox}

    \hypertarget{c-test-de-lalgorithme-de-horner-et-muxe9thode-des-invariants}{%
\subsubsection{\texorpdfstring{\textbf{1.c) Test de l'Algorithme de
Horner et méthode des
invariants}}{1.c) Test de l'Algorithme de Horner et méthode des invariants}}\label{c-test-de-lalgorithme-de-horner-et-muxe9thode-des-invariants}}

    \begin{tcolorbox}[breakable, size=fbox, boxrule=1pt, pad at break*=1mm,colback=cellbackground, colframe=cellborder]
\prompt{In}{incolor}{4}{\boxspacing}
\begin{Verbatim}[commandchars=\\\{\}]
\PY{n}{c} \PY{o}{=} \PY{n}{random}\PY{o}{.}\PY{n}{uniform}\PY{p}{(}\PY{l+m+mi}{0}\PY{p}{,}\PY{l+m+mi}{1}\PY{p}{)} \PY{c+c1}{\PYZsh{} On génère un nbre aléatoire entre 0 et 1 pour tester notre fonction}
\PY{n}{A} \PY{o}{=} \PY{p}{[}\PY{o}{\PYZhy{}}\PY{n}{k\PYZus{}a}\PY{o}{*}\PY{n}{k\PYZus{}e}\PY{p}{,}\PY{o}{\PYZhy{}}\PY{p}{(}\PY{n}{k\PYZus{}e}\PY{o}{+}\PY{n}{c}\PY{o}{*}\PY{n}{k\PYZus{}a}\PY{p}{)}\PY{p}{,} \PY{n}{k\PYZus{}a}\PY{p}{,}\PY{l+m+mi}{1}\PY{p}{]}
\end{Verbatim}
\end{tcolorbox}

    \begin{tcolorbox}[breakable, size=fbox, boxrule=1pt, pad at break*=1mm,colback=cellbackground, colframe=cellborder]
\prompt{In}{incolor}{5}{\boxspacing}
\begin{Verbatim}[commandchars=\\\{\}]
\PY{n}{horner}\PY{p}{(}\PY{n}{A}\PY{p}{,}\PY{l+m+mi}{0}\PY{p}{)}\PY{o}{==} \PY{o}{\PYZhy{}}\PY{n}{k\PYZus{}a}\PY{o}{*}\PY{n}{k\PYZus{}e} \PY{c+c1}{\PYZsh{} on teste l\PYZsq{}image de 0}
\PY{n}{horner}\PY{p}{(}\PY{n}{A}\PY{p}{,}\PY{l+m+mi}{1}\PY{p}{)}\PY{o}{==} \PY{n+nb}{sum}\PY{p}{(}\PY{n}{A}\PY{p}{)}   \PY{c+c1}{\PYZsh{} on teste l\PYZsq{}image de 1}
\end{Verbatim}
\end{tcolorbox}

            \begin{tcolorbox}[breakable, size=fbox, boxrule=.5pt, pad at break*=1mm, opacityfill=0]
\prompt{Out}{outcolor}{5}{\boxspacing}
\begin{Verbatim}[commandchars=\\\{\}]
True
\end{Verbatim}
\end{tcolorbox}
        
    \hypertarget{muxe9thode-des-invariants}{%
\subsubsection{\texorpdfstring{\textbf{Méthode des
invariants}}{Méthode des invariants}}\label{muxe9thode-des-invariants}}

\textbf{Un invariant de boucle est une propriéte Q qui vérifie:} -
\textbf{Est vraie avant l'entrée dans la boucle.} - \textbf{Est toujours
vraie après chaque itération de la boucle.} - \textbf{Vaut la propriéte
P desiree a la fin de l'exécution de la boucle.}

\hypertarget{cas-de-lalgorithme-de-horner}{%
\paragraph{\texorpdfstring{\textbf{Cas de l'Algorithme de
Horner}}{Cas de l'Algorithme de Horner}}\label{cas-de-lalgorithme-de-horner}}

Notons \(A_n=\{a_0,a_1,...,a_n\}\) et \(P_i = horner(A_i,x)\)
\textgreater{} \textbf{Invariant de boucle:} A chaque itération i, on a:
\(P_i=A_nx^{n-i}+A{_{n-i-1}}x^{n-1}+...+A_{i+1}x+A_i=\sum_{j=0}^{n-i}A_{n-j}x^{n-i-j}\)

\hypertarget{preuve}{%
\subparagraph{\texorpdfstring{\textbf{Preuve}:}{Preuve:}}\label{preuve}}

\begin{itemize}
\tightlist
\item
  \textbf{Avant la boucle}, \(i = n\) \textgreater{} On a
  \(P_i = \sum_{j=0}^{n-n}A_{n-j}x^{n-i-j}=A_n\)
\item
  Montrons que à chaque prochaine itération \(inew\), on a
  \(horner(A_{inew},x)\Rightarrow horner(A_{iold},x)\hspace{10px}\) où
  \$\hspace{10px}iold \$ est l'itération précédente
\end{itemize}

\begin{quote}
\(P_{iold}= (A_nx^{n-iold}+A_{n-1}x^{n-iold-1}+...+A_{iold})x+A_{iold-1}\)
\(= A_nx^{n-iold+1}+A_{n-1}x^{n-iold}+...+A_{iold}x+A_{iold-1}\)
\(= A_nx^{n-inew}+A_{n-1}x^{n-inew-1}+...+A_{inew+1}x+A_{inew}\)
\(=\sum_{j=0}^{n-inew}A_{n-j}x^{n-inew-j}\) \(=P_{inew}\)
\end{quote}

\begin{quote}
Autrement dit, à chaque itération i, on a
\(P_i = horner(A_i,x) \Rightarrow P_{i-1} = horner(A_{i-1},x)\)
\end{quote}

\begin{itemize}
\tightlist
\item
  Après la boucle, \(i=0\) et on a bien
  \(P_{0}(X)=a_nx^{n}+a_{n-1}x^{i-1}+...+a_{1}x+a_0\)
\end{itemize}

    \textbf{Les étapes de l'algorithme de Horner sont les suivantes:}

\$ \textbackslash{} \left. \textbackslash{}begin\{array\}\{ll\}
P\_\{n\}(X)=a\_nX \textbackslash{}
P\_\{n-1\}(X)=P\_\{n\}X+a\_\{n-1\}=a\_\{n\}x+a\_\{n-1\} \textbackslash{}
\vdots \textbackslash{}
P\_\{1\}(X)=P\_\{2\}X+a\_\{1\}=a\_nx\textsuperscript{\{n-1\}+a\_\{n-2\}x}\{n-2\}+\ldots{}+a\_\{2\}x+a\_1\textbackslash{}
P\_\{0\}(X)=P\_\{1\}X+a\_\{0\}=a\_nx\textsuperscript{\{n\}+a\_\{n-1\}x}\{n-1\}+\ldots{}+a\_\{1\}x+a\_0\textbackslash{}
\textbackslash{}end\{array\} \right \} n-multiplications \$

n décroit jusqu'a 0 et à la fin on a bien le résultat désiré
\(P(X)= a_nx^{n}+a_{n-1}x^{n-1}+...+a_{1}x+a_0\) 

    \hypertarget{d-valeurs-de-c0-en-fonction-de-k_e-et-k_a-pour-que-h_3oleqslant-1}{%
\subsubsection{\texorpdfstring{\textbf{1.d) Valeurs de \(c>0\) en
fonction de \(k_e\) et \(k_a\) pour que
\([H_3O^+]\leqslant 1\)}}{1.d) Valeurs de c\textgreater{}0 en fonction de k\_e et k\_a pour que {[}H\_3O\^{}+{]}\textbackslash{}leqslant 1}}\label{d-valeurs-de-c0-en-fonction-de-k_e-et-k_a-pour-que-h_3oleqslant-1}}

D'après la question (1-a), on a :

\([A^-]=[H_3O^+]-\frac{k_e}{[H_3O^+]} \hspace{40px}(*)\)

et \([A^-]=c-\frac{1}{k_a}([H_3O^+]^2-k_e)\hspace{40px}(**)\)

\(\hspace{40px}[H_3O+]\leqslant 1 \iff \frac{1}{[H_3O+]}\geq 1 \iff \frac{k_e}{[H_3O+]}\geq k_e \iff -\frac{k_e}{[H_3O+]}\leqslant -k_e \iff [H_3O^+]-\frac{k_e}{[H_3O+]}\leqslant -k_e+[H_3O^+]\leqslant 1-k_e\)

Ainsi,on a
\(\hspace{40px}[H_3O^+]\leqslant 1 \iff [H_3O^+]-\frac{k_e}{[H_3O+]}\leqslant -k_e+[H_3O^+]\leqslant 1-k_e\)

\((*)\) devient \(\hspace{40px}[A^-]\leqslant 1-k_e\)

Dans \((**)\), on
obtient:\(\hspace{40px}c-\frac{1}{k_a}([H_3O^+]^2-k_e)\leqslant 1-k_e\)

\(\hspace{135px}\iff c\leqslant 1-k_e+\frac{1}{k_a}[H_3O^+]^2-\frac{k_e}{k_a}\)

En majorant \([H_3O^+]\) par 1, on obtient
\(c\leqslant 1-k_e+\frac{1}{k_a}-\frac{k_e}{k_a}=\frac{(1-k_e)(1+k_a)}{k_a}\)

On a finalement: \textbf{\(c \in ]0,\frac{(1-k_e)(1+k_a)}{k_a}]\)}

    \hypertarget{e-repruxe9sentation-du-polynuxf4me-p-sur-00.005-pour-c-in1.-0.10.0110-12}{%
\subsubsection{\texorpdfstring{\textbf{e) Représentation du polynôme p
sur {[}0;0.005{]} pour c
\(\in\)\{\({1., 0.1,0.01,10^{-12}}\)\}}}{e) Représentation du polynôme p sur {[}0;0.005{]} pour c \textbackslash{}in\{\{1., 0.1,0.01,10\^{}\{-12\}\}\}}}\label{e-repruxe9sentation-du-polynuxf4me-p-sur-00.005-pour-c-in1.-0.10.0110-12}}

    \begin{tcolorbox}[breakable, size=fbox, boxrule=1pt, pad at break*=1mm,colback=cellbackground, colframe=cellborder]
\prompt{In}{incolor}{6}{\boxspacing}
\begin{Verbatim}[commandchars=\\\{\}]
\PY{n}{C} \PY{o}{=} \PY{n}{np}\PY{o}{.}\PY{n}{array}\PY{p}{(}\PY{p}{[}\PY{l+m+mf}{1.}\PY{p}{,}\PY{l+m+mf}{0.1}\PY{p}{,} \PY{l+m+mf}{0.01}\PY{p}{,} \PY{l+m+mi}{10}\PY{o}{*}\PY{o}{*}\PY{p}{(}\PY{o}{\PYZhy{}}\PY{l+m+mi}{12}\PY{p}{)}\PY{p}{]}\PY{p}{)}
\PY{n}{coefs} \PY{o}{=} \PY{p}{[}\PY{p}{]} \PY{c+c1}{\PYZsh{} vecteur contenant les coefficients du polynôme pour les différentes valeurs de C}
\PY{k}{for} \PY{n}{i} \PY{o+ow}{in} \PY{n}{C}\PY{p}{:}
    \PY{n}{res} \PY{o}{=} \PY{n}{np}\PY{o}{.}\PY{n}{array}\PY{p}{(}\PY{p}{[}\PY{o}{\PYZhy{}}\PY{n}{k\PYZus{}a}\PY{o}{*}\PY{n}{k\PYZus{}e}\PY{p}{,} \PY{o}{\PYZhy{}}\PY{p}{(}\PY{n}{k\PYZus{}e}\PY{o}{+}\PY{n}{i}\PY{o}{*}\PY{n}{k\PYZus{}a}\PY{p}{)}\PY{p}{,} \PY{n}{k\PYZus{}a}\PY{p}{,} \PY{l+m+mi}{1}\PY{p}{]}\PY{p}{)}
    \PY{n}{coefs}\PY{o}{.}\PY{n}{append}\PY{p}{(}\PY{n}{res}\PY{p}{)}
    \PY{n}{coefs}
\end{Verbatim}
\end{tcolorbox}

    \begin{tcolorbox}[breakable, size=fbox, boxrule=1pt, pad at break*=1mm,colback=cellbackground, colframe=cellborder]
\prompt{In}{incolor}{7}{\boxspacing}
\begin{Verbatim}[commandchars=\\\{\}]
\PY{n}{coefs}
\end{Verbatim}
\end{tcolorbox}

            \begin{tcolorbox}[breakable, size=fbox, boxrule=.5pt, pad at break*=1mm, opacityfill=0]
\prompt{Out}{outcolor}{7}{\boxspacing}
\begin{Verbatim}[commandchars=\\\{\}]
[array([-1.82088e-19, -1.80000e-05,  1.80000e-05,  1.00000e+00]),
 array([-1.82088000e-19, -1.80000001e-06,  1.80000000e-05,  1.00000000e+00]),
 array([-1.8208800e-19, -1.8000001e-07,  1.8000000e-05,  1.0000000e+00]),
 array([-1.82088e-19, -1.01340e-14,  1.80000e-05,  1.00000e+00])]
\end{Verbatim}
\end{tcolorbox}
        
    \begin{tcolorbox}[breakable, size=fbox, boxrule=1pt, pad at break*=1mm,colback=cellbackground, colframe=cellborder]
\prompt{In}{incolor}{8}{\boxspacing}
\begin{Verbatim}[commandchars=\\\{\}]
\PY{n}{x} \PY{o}{=} \PY{n}{np}\PY{o}{.}\PY{n}{linspace}\PY{p}{(}\PY{l+m+mi}{0}\PY{p}{,}\PY{l+m+mf}{0.005}\PY{p}{,} \PY{l+m+mi}{100}\PY{p}{)}
\PY{n}{px1} \PY{o}{=} \PY{p}{[}\PY{p}{]}\PY{p}{;}\PY{n}{px2} \PY{o}{=} \PY{p}{[}\PY{p}{]}\PY{p}{;}\PY{n}{px3} \PY{o}{=} \PY{p}{[}\PY{p}{]}\PY{p}{;}\PY{n}{px4} \PY{o}{=} \PY{p}{[}\PY{p}{]}\PY{p}{;} \PY{c+c1}{\PYZsh{} initialisation des vecteurs  qui recevront les valeurs des polynômes}
\PY{k}{for} \PY{n}{i} \PY{o+ow}{in} \PY{n+nb}{range}\PY{p}{(}\PY{n+nb}{len}\PY{p}{(}\PY{n}{x}\PY{p}{)}\PY{p}{)}\PY{p}{:}
    \PY{n}{px1}\PY{o}{.}\PY{n}{append}\PY{p}{(}\PY{n}{horner}\PY{p}{(}\PY{n}{coefs}\PY{p}{[}\PY{l+m+mi}{0}\PY{p}{]}\PY{p}{,}\PY{n}{x}\PY{p}{[}\PY{n}{i}\PY{p}{]}\PY{p}{)}\PY{p}{)}
    \PY{n}{px2}\PY{o}{.}\PY{n}{append}\PY{p}{(}\PY{n}{horner}\PY{p}{(}\PY{n}{coefs}\PY{p}{[}\PY{l+m+mi}{1}\PY{p}{]}\PY{p}{,}\PY{n}{x}\PY{p}{[}\PY{n}{i}\PY{p}{]}\PY{p}{)}\PY{p}{)}
    \PY{n}{px3}\PY{o}{.}\PY{n}{append}\PY{p}{(}\PY{n}{horner}\PY{p}{(}\PY{n}{coefs}\PY{p}{[}\PY{l+m+mi}{2}\PY{p}{]}\PY{p}{,}\PY{n}{x}\PY{p}{[}\PY{n}{i}\PY{p}{]}\PY{p}{)}\PY{p}{)}
    \PY{n}{px4}\PY{o}{.}\PY{n}{append}\PY{p}{(}\PY{n}{horner}\PY{p}{(}\PY{n}{coefs}\PY{p}{[}\PY{l+m+mi}{3}\PY{p}{]}\PY{p}{,}\PY{n}{x}\PY{p}{[}\PY{n}{i}\PY{p}{]}\PY{p}{)}\PY{p}{)}

\PY{n}{px} \PY{o}{=} \PY{p}{[}\PY{n}{px1}\PY{p}{,}\PY{n}{px2}\PY{p}{,}\PY{n}{px3}\PY{p}{,}\PY{n}{px4}\PY{p}{]}
\PY{n}{colors} \PY{o}{=} \PY{p}{[}\PY{l+s+s1}{\PYZsq{}}\PY{l+s+s1}{b}\PY{l+s+s1}{\PYZsq{}}\PY{p}{,}\PY{l+s+s1}{\PYZsq{}}\PY{l+s+s1}{g}\PY{l+s+s1}{\PYZsq{}}\PY{p}{,}\PY{l+s+s1}{\PYZsq{}}\PY{l+s+s1}{r}\PY{l+s+s1}{\PYZsq{}}\PY{p}{,}\PY{l+s+s1}{\PYZsq{}}\PY{l+s+s1}{y}\PY{l+s+s1}{\PYZsq{}}\PY{p}{]}
\PY{n}{labs} \PY{o}{=} \PY{p}{[}\PY{l+s+s2}{\PYZdq{}}\PY{l+s+s2}{c=1.}\PY{l+s+s2}{\PYZdq{}}\PY{p}{,}\PY{l+s+s2}{\PYZdq{}}\PY{l+s+s2}{c=0.1}\PY{l+s+s2}{\PYZdq{}}\PY{p}{,} \PY{l+s+s2}{\PYZdq{}}\PY{l+s+s2}{c=0.01}\PY{l+s+s2}{\PYZdq{}}\PY{p}{,} \PY{l+s+s2}{\PYZdq{}}\PY{l+s+s2}{c=\PYZdl{}10\PYZca{}}\PY{l+s+s2}{\PYZob{}}\PY{l+s+s2}{\PYZhy{}12\PYZcb{}\PYZdl{}}\PY{l+s+s2}{\PYZdq{}}\PY{p}{]}

\PY{k}{for} \PY{n}{i} \PY{o+ow}{in} \PY{n+nb}{range}\PY{p}{(}\PY{l+m+mi}{0}\PY{p}{,}\PY{l+m+mi}{4}\PY{p}{)}\PY{p}{:}
    \PY{n}{plt}\PY{o}{.}\PY{n}{plot}\PY{p}{(}\PY{n}{x}\PY{p}{,}\PY{n}{px}\PY{p}{[}\PY{n}{i}\PY{p}{]}\PY{p}{,}\PY{n}{color}\PY{o}{=}\PY{n}{colors}\PY{p}{[}\PY{n}{i}\PY{p}{]}\PY{p}{,}\PY{n}{label}\PY{o}{=}\PY{n}{labs}\PY{p}{[}\PY{n}{i}\PY{p}{]}\PY{p}{)}
    
\PY{n}{plt}\PY{o}{.}\PY{n}{xlabel}\PY{p}{(}\PY{l+s+s1}{\PYZsq{}}\PY{l+s+s1}{x}\PY{l+s+s1}{\PYZsq{}}\PY{p}{)}
\PY{n}{plt}\PY{o}{.}\PY{n}{ylabel}\PY{p}{(}\PY{l+s+s1}{\PYZsq{}}\PY{l+s+s1}{P(x)}\PY{l+s+s1}{\PYZsq{}}\PY{p}{)}
\PY{n}{plt}\PY{o}{.}\PY{n}{grid}\PY{p}{(}\PY{p}{)}
\PY{n}{plt}\PY{o}{.}\PY{n}{legend}\PY{p}{(}\PY{p}{)}
\PY{n}{plt}\PY{o}{.}\PY{n}{show}\PY{p}{(}\PY{p}{)}
\end{Verbatim}
\end{tcolorbox}

    \begin{center}
    \adjustimage{max size={0.9\linewidth}{0.9\paperheight}}{output_21_0.png}
    \end{center}
    { \hspace*{\fill} \\}
    
    \hypertarget{f-courbe-du-ph-en-fonction-de--log_10c}{%
\subsubsection{\texorpdfstring{\textbf{1.f) Courbe du PH en fonction de
\({-log_{10}c}\)}}{1.f) Courbe du PH en fonction de \{-log\_\{10\}c\}}}\label{f-courbe-du-ph-en-fonction-de--log_10c}}

    A partir de la suite \(x_{n+1}=x_n+h\frac{f'(x_n)}{f(x_n)}\), on écris
une routine qui, avec un pas constant h, détermine le point de la
prochaine itération.

    \begin{tcolorbox}[breakable, size=fbox, boxrule=1pt, pad at break*=1mm,colback=cellbackground, colframe=cellborder]
\prompt{In}{incolor}{9}{\boxspacing}
\begin{Verbatim}[commandchars=\\\{\}]
\PY{k}{def} \PY{n+nf}{nextPoint}\PY{p}{(}\PY{n}{x0}\PY{p}{,}\PY{n}{fx0}\PY{p}{,}\PY{n}{fprim\PYZus{}x0}\PY{p}{,}\PY{n}{h}\PY{o}{=}\PY{l+m+mi}{10}\PY{o}{*}\PY{o}{*}\PY{p}{(}\PY{o}{\PYZhy{}}\PY{l+m+mi}{10}\PY{p}{)}\PY{p}{)}\PY{p}{:} \PY{c+c1}{\PYZsh{} fonctionne dans notre cas à partir de h = 10**(\PYZhy{}3)}
    \PY{n}{xnext} \PY{o}{=} \PY{n}{x0}\PY{o}{+}\PY{n}{h}\PY{o}{*}\PY{n}{fprim\PYZus{}x0}\PY{o}{/}\PY{n}{fx0}
    \PY{k}{return} \PY{n}{xnext}
\end{Verbatim}
\end{tcolorbox}

    \begin{tcolorbox}[breakable, size=fbox, boxrule=1pt, pad at break*=1mm,colback=cellbackground, colframe=cellborder]
\prompt{In}{incolor}{10}{\boxspacing}
\begin{Verbatim}[commandchars=\\\{\}]
\PY{n}{C} \PY{o}{=} \PY{n}{np}\PY{o}{.}\PY{n}{linspace}\PY{p}{(}\PY{l+m+mi}{0}\PY{p}{,}\PY{l+m+mi}{1}\PY{p}{,} \PY{l+m+mi}{100}\PY{p}{)}
\PY{n}{coef3} \PY{o}{=} \PY{p}{[}\PY{p}{]} \PY{c+c1}{\PYZsh{} vecteur contenant le 3e coéfficient du polynôme pour les différentes valeurs de C}
\PY{k}{for} \PY{n}{i} \PY{o+ow}{in} \PY{n}{C}\PY{p}{:}
    \PY{n}{coef3}\PY{o}{.}\PY{n}{append}\PY{p}{(}\PY{p}{(}\PY{n}{k\PYZus{}e}\PY{o}{+}\PY{n}{i}\PY{o}{*}\PY{n}{k\PYZus{}a}\PY{p}{)}\PY{p}{)}
    \PY{n}{coef3}

\PY{n}{racines} \PY{o}{=} \PY{p}{[}\PY{p}{]} 
\PY{n}{n}\PY{o}{=}\PY{n+nb}{len}\PY{p}{(}\PY{n}{coef3}\PY{p}{)}\PY{o}{\PYZhy{}}\PY{l+m+mi}{1}
\PY{n}{xnext} \PY{o}{=} \PY{l+m+mi}{1}\PY{o}{+}\PY{l+m+mf}{1.0e\PYZhy{}15} \PY{c+c1}{\PYZsh{} on initialise par une valeur non nulle très proche de zero}
\PY{k}{for} \PY{n}{i} \PY{o+ow}{in} \PY{n+nb}{range}\PY{p}{(}\PY{n}{n}\PY{p}{,}\PY{o}{\PYZhy{}}\PY{l+m+mi}{1}\PY{p}{,}\PY{o}{\PYZhy{}}\PY{l+m+mi}{1}\PY{p}{)}\PY{p}{:}
    \PY{n}{f} \PY{o}{=} \PY{k}{lambda} \PY{n}{x}\PY{p}{:} \PY{n}{x}\PY{o}{*}\PY{o}{*}\PY{l+m+mi}{3}\PY{o}{+}\PY{n}{k\PYZus{}a}\PY{o}{*}\PY{n}{x}\PY{o}{*}\PY{o}{*}\PY{l+m+mi}{2}\PY{o}{\PYZhy{}}\PY{n}{coef3}\PY{p}{[}\PY{n}{i}\PY{p}{]}\PY{o}{*}\PY{n}{x}\PY{o}{\PYZhy{}}\PY{n}{k\PYZus{}a}\PY{o}{*}\PY{n}{k\PYZus{}e}
    \PY{n}{fp} \PY{o}{=} \PY{k}{lambda} \PY{n}{x}\PY{p}{:} \PY{l+m+mi}{3}\PY{o}{*}\PY{p}{(}\PY{n}{x}\PY{o}{*}\PY{o}{*}\PY{l+m+mi}{2}\PY{p}{)}\PY{o}{\PYZhy{}}\PY{l+m+mi}{2}\PY{o}{*}\PY{n}{k\PYZus{}a}\PY{o}{*}\PY{n}{x}\PY{o}{\PYZhy{}}\PY{n}{coef3}\PY{p}{[}\PY{n}{i}\PY{p}{]}
    \PY{n}{xnext} \PY{o}{=} \PY{n}{nextPoint}\PY{p}{(}\PY{n}{xnext}\PY{p}{,}\PY{n}{f}\PY{p}{(}\PY{n}{xnext}\PY{p}{)}\PY{p}{,}\PY{n}{fp}\PY{p}{(}\PY{n}{xnext}\PY{p}{)}\PY{p}{)}
    \PY{n}{res} \PY{o}{=} \PY{n}{brentq}\PY{p}{(}\PY{n}{f}\PY{p}{,} \PY{l+m+mi}{0}\PY{p}{,} \PY{n}{xnext}\PY{p}{)} 
    \PY{n}{racines}\PY{o}{.}\PY{n}{append}\PY{p}{(}\PY{n}{res}\PY{p}{)}
\end{Verbatim}
\end{tcolorbox}

    \begin{tcolorbox}[breakable, size=fbox, boxrule=1pt, pad at break*=1mm,colback=cellbackground, colframe=cellborder]
\prompt{In}{incolor}{11}{\boxspacing}
\begin{Verbatim}[commandchars=\\\{\}]
\PY{n}{x} \PY{o}{=} \PY{p}{[}\PY{o}{\PYZhy{}}\PY{n}{log10}\PY{p}{(}\PY{n}{r}\PY{p}{)} \PY{k}{for} \PY{n}{r} \PY{o+ow}{in} \PY{n}{racines}\PY{p}{]}
\PY{n}{plt}\PY{o}{.}\PY{n}{plot}\PY{p}{(}\PY{n}{x}\PY{p}{)}
\PY{n}{plt}\PY{o}{.}\PY{n}{title}\PY{p}{(}\PY{l+s+s2}{\PYZdq{}}\PY{l+s+s2}{Courbe du PH en fonction de \PYZdl{}\PYZhy{}log\PYZus{}}\PY{l+s+si}{\PYZob{}10\PYZcb{}}\PY{l+s+s2}{(c)\PYZdl{}}\PY{l+s+s2}{\PYZdq{}}\PY{p}{)}
\PY{n}{plt}\PY{o}{.}\PY{n}{xlabel}\PY{p}{(}\PY{l+s+s2}{\PYZdq{}}\PY{l+s+s2}{\PYZdl{}\PYZhy{}log\PYZus{}}\PY{l+s+si}{\PYZob{}10\PYZcb{}}\PY{l+s+s2}{(c)\PYZdl{}}\PY{l+s+s2}{\PYZdq{}}\PY{p}{)}
\PY{n}{plt}\PY{o}{.}\PY{n}{ylabel}\PY{p}{(}\PY{l+s+s2}{\PYZdq{}}\PY{l+s+s2}{PH}\PY{l+s+s2}{\PYZdq{}}\PY{p}{)}
\PY{n}{plt}\PY{o}{.}\PY{n}{grid}\PY{p}{(}\PY{p}{)}
\PY{n}{plt}\PY{o}{.}\PY{n}{show}\PY{p}{(}\PY{p}{)}
\end{Verbatim}
\end{tcolorbox}

    \begin{center}
    \adjustimage{max size={0.9\linewidth}{0.9\paperheight}}{output_26_0.png}
    \end{center}
    { \hspace*{\fill} \\}
    
    \textbf{Explications:}

\begin{quote}
On remarque que PH est une fonction croissante de \(-log_{10}(c)\). Ce
qui signifie que lorsque \(-log_{10}(c)\) augmente(ie. c diminue), le PH
augmente aussi.
\end{quote}

\begin{quote}
Ce qui \textbf{n'est pas logique d'un point de vue chimique} puisque
pour un acide, le PH décroit en fonction de sa concentration. En effet,
plus la concentration en ions oxonium est faible, plus le pH augmente et
plus la solution est basique. Inversement, plus la concentration en ions
oxonium est importante, plus le pH diminue et plus la solution est
acide.
\end{quote}

    \hypertarget{g-preuve-des-propriuxe9tuxe9s-constatuxe9es-graphiquement}{%
\subsubsection{\texorpdfstring{\textbf{1.g) Preuve des propriétés
constatées
graphiquement}}{1.g) Preuve des propriétés constatées graphiquement}}\label{g-preuve-des-propriuxe9tuxe9s-constatuxe9es-graphiquement}}

\begin{itemize}
\tightlist
\item
  \textbf{Continuité}:
\end{itemize}

Nous avons montré à la question (a) que pour tout réel
\(c \in ]0,c_1]\), la fonction \(p(c,.)\) est Contractante sur
\([0,x_1]\). Ainsi d'apres le théorème du point fixe sur \(\mathbb{R}\),
il existe une application \(\gamma :]0,c_1]\longmapsto [0,x_1]\) telle
que pour tout \(c\in ]0,c_1], \gamma (c)\) soit l'unique point fixe de
\(p(c,.)\).

De plus, comme la fonction \(c \longmapsto p(c,x)\) est continue pour
tout \(x\in [0,1]\), on en déduit que la fonction \(\gamma\) est
également continue .

\begin{itemize}
\item
  \textbf{Monotonie}:
\item
  \textbf{Dérivabilité}:
\end{itemize}

    \hypertarget{preuve-que-ph-longrightarrow--log_10sqrtk_eapprox-7-lorsque-clongrightarrow-0}{%
\subsubsection{\texorpdfstring{\textbf{Preuve que pH
\(\longrightarrow -log_{10}\sqrt{k_e}\approx 7\) lorsque
\(c\longrightarrow 0\)}}{Preuve que pH \textbackslash{}longrightarrow -log\_\{10\}\textbackslash{}sqrt\{k\_e\}\textbackslash{}approx 7 lorsque c\textbackslash{}longrightarrow 0}}\label{preuve-que-ph-longrightarrow--log_10sqrtk_eapprox-7-lorsque-clongrightarrow-0}}

    Lorsque \(c\longrightarrow 0\), notre équation de départ devient
\(X^3+k_{a}X^2-k_{e}X+k_ak_e=0\).

Posons \(Q(X)=X^3+k_{a}X^2-k_{e}X+k_ak_e\), on remarque que
\(Q(-k_a)=0\)

Grace à cette racine, on obtient la forme factorisée suivante :
\(Q(X)= (X+k_a)(X^2-k_e)\)

Donc \(Q(X)=0 \iff X+k_a=0\) ou \(X^2-k_e=0\) \(\iff X=-k_a\) ou
\(X= \pm \sqrt{k_e}\)

Mais \(X>0\), on a donc \(X = \sqrt(k_e)\)

Ainsi,lorsque \(c\longrightarrow 0\), on a
\([H_3O^+] \longrightarrow \sqrt(k_e)\).

Et comme \(pH = -log_{10}[H_3O^+]\)

On en déduit que \(pH \longrightarrow -log_{10}\sqrt{k_e}\) lorsque
\(c\longrightarrow 0\)

    \hypertarget{question-2-adapdation-de-la-muxe9thode-du-point-fixe}{%
\subsection{\texorpdfstring{\textbf{Question 2: Adapdation de la méthode
du point
fixe}}{Question 2: Adapdation de la méthode du point fixe}}\label{question-2-adapdation-de-la-muxe9thode-du-point-fixe}}

    \begin{itemize}
\tightlist
\item
  \textbf{Montrons que la suite \((x_n)_{n\in \mathbb{N}}\) est de
  Cauchy:}
\end{itemize}

Soit \(x_0 \in B\), comme B est stable par \(\varphi(c,.)\) par
défintiion de \(\varphi(c,.)\), on peut poser
\(x_n = \varphi(c,x_{n-1})\) pour tout \(n \in \mathbb{N}^*\)

On a alors:

\(\left\lVert x_{n+1}-x_n \right\rVert = \left\lVert \varphi(c,x_n) -\varphi(c,x_{n-1})\right\rVert \leqslant L\left\lVert x_n-x_{n-1}\right\rVert\)

On obtient par récurrence sur n que

\(\left\lVert \varphi(c,x_{n+1}) -\varphi(c,x_{n})\right\rVert \leqslant L^n \left\lVert x_1-x_{0}\right\rVert\)

Ainsi, pour tout \(n\in \mathbb{N}\) et \(k \in \mathbb{N}^*\), on a :

\(\left\lVert x_{n+k} -x_{n})\right\rVert \leqslant \sum_{i=0}^{k-1} \left\lVert x_{n+i+1}-x_{n+i}\right\rVert\).

\(\hspace{100px}\leqslant \sum_{i=0}^{k-1} L^{n+i}\left\lVert x_{1}-x_{0}\right\rVert\)

\(\hspace{100px}\leqslant \left\lVert x_{1}-x_{0}\right\rVert \sum_{i=0}^{k-1} L^{n+i}\)

\(\hspace{100px}\leqslant L^n\frac{1-L^k}{1-L}\left\lVert x_{1}-x_{0}\right\rVert \hspace{10px}\)

\(\hspace{100px}\leqslant \frac{L^n}{1-L}\left\lVert x_{1}-x_{0}\right\rVert \longrightarrow 0\)
quand n \(\longrightarrow +\infty\) car \(0<L<1\)

Ainsi la suite \((x_n)_{n \in \mathbb{N}}\) est de Cauchy.

\begin{itemize}
\tightlist
\item
  \textbf{Convergence:}
\end{itemize}

Comme \(\mathbb{R}^n\) est complet, alors il existe
\(x^* \in \mathbb{R}\) tel que \(x_n \longrightarrow x^*\).

De plus comme B est fermé, cette limite \(x^* \in B\) et
\(\forall n\in \mathbb{N}\), on a \(x_{n+1}=\varphi(c,x_n)\)

Par ailleurs, en utilisant la continuité de \(\varphi(c,.)\) et en
passant à la limite, on a
\(x^*=lim_{n \longrightarrow +\infty}x_{n+1}=lim_{n \longrightarrow +\infty} \varphi (c,x_n)=\varphi (c, lim_{n \longrightarrow +\infty} x_n)=\varphi (c,x^*)\)

\begin{itemize}
\tightlist
\item
  \textbf{Unicité:}
\end{itemize}

Soit \(L\in ]0,1[\) tel que:

\(\forall c\in I, \forall x_1,x_2 \in B \left\lVert \varphi(c,x_1) -\varphi(c,x_2)\right\rVert \leqslant L\left\lVert x_1-x_2\right\rVert\)

Supposons que x et y sont deux points fixes de \(\varphi (c,.)\), alors
on a:

\(\left\lVert x-y \right\lvert=\left\lVert \varphi (c,x)-\varphi(c,y) \right\lvert \leqslant L\left\lVert x-y \right\lvert\)
ce qui implique \(\left\lVert x-y \right\lvert(1-L) \leqslant 0\)

Mais on a \(1-L>0\) car \(L\in ]0,1[\)

Donc \(\left\lVert x-y \right\lvert=0\) et donc x=y. D'ou l'unicité de
\(x^*\)

On peut ainsi définir la fonction \(\gamma:I\longrightarrow B\)

\(c\longmapsto\) l'unique \(x^*\) tel que \(x^*=\varphi (c,x^*)\)

    \hypertarget{question-3-montrons-que-la-fonction-varphi-vuxe9rifie-les-hypothuxe8ses-du-thuxe9oruxe8me-du-point-fixe}{%
\subsection{\texorpdfstring{\textbf{Question 3: Montrons que la fonction
\(\varphi\) vérifie les hypothèses du théorème du point
fixe}}{Question 3: Montrons que la fonction \textbackslash{}varphi vérifie les hypothèses du théorème du point fixe}}\label{question-3-montrons-que-la-fonction-varphi-vuxe9rifie-les-hypothuxe8ses-du-thuxe9oruxe8me-du-point-fixe}}

    On sait que pour tout \(c\in I\), la fonction \(\varphi(c,.)\) est
continue sur B

De plus, l'hypothèse (a) nous dit que \(\partial _x \varphi (c,x)\)
existe en tout point \(c\in I\) et \(x\in B\)

Ces deux hypothèses montrent que pour tout \(c\in I\) la fonction
\textbf{\(\varphi(c,.)\) est de classe \(C^1\) sur B}

Sous ces hypothèses, le \textbf{Théorème des Accroissement Finis} nous
assure que pour tout \(x_1, x_2\in B\):

\(\left\lVert \varphi (c,x_1)-\varphi(c,x_2) \right\lVert \leqslant \left\lVert x_1-x_2 \right\lVert \sup_{x\in intB}\left\lVert \partial _x \varphi(c,x) \right\lVert\)

\(\hspace{150px}\leqslant \left\lVert x_1-x_2 \right\lVert \sup_{c\in I}(\sup_{x\in intB}\left\lVert \partial _x \varphi(c,x) \right\lVert)\)

\$\hspace{150px}\textless{} \left\lVert x\_1-x\_2
\right\lVert  \hspace{50px} \$ Car
\(\sup_{c\in I}(\sup_{x\in intB}\left\lVert \partial _x \varphi(c,x) \right\lVert)<1\)
d'après l'hypothèse (b)

    \hypertarget{question-4-preuve-du-thuxe9oruxe8me-des-fonctions-implicites}{%
\subsection{\texorpdfstring{\textbf{Question 4: Preuve du Théorème des
fonctions
implicites}}{Question 4: Preuve du Théorème des fonctions implicites}}\label{question-4-preuve-du-thuxe9oruxe8me-des-fonctions-implicites}}

    \textbf{Etape 1:} Montrons que la fonction \(\varphi\) est
différentiable dans un voisinage approprié de \((c_0,x_0)\).

\begin{itemize}
\tightlist
\item
  \textbf{Montrons que la fonction \(\gamma = \varphi (c,.)\) est
  Lipschitzienne}
\end{itemize}

Comme \(\frac{\partial f}{\partial x}(c,.)\) est continue en \(c_0\),
alors \(\frac{\partial \varphi}{\partial x}(c,.)\) l'est aussi.

Soit \(q \in ]0,1[\), on a ,par définition de la continuité de
\(\frac{\partial \varphi}{\partial x}\) au point \(c_0\),qu'il existe
\(r>0\) et \(R>0\) tel que:

\(\forall x \in B, \left\lVert c-c_0 \right\lVert \leqslant r\),\(\left\lVert x-c_0 \right\lVert \leqslant R \implies \left\lVert \frac{\partial \varphi}{\partial x}(c,x)-\frac{\partial \varphi}{\partial x}(c_0,c_0)\right\lVert \leqslant q \hspace{50px}\)
\((i)\)

Par définition de la continuité de \(\gamma\) au point \(c_0\),on a que:

Pour tout \$\epsilon \textgreater{}0, \$il existe \(\delta >0\) tel que:
\(\left\lVert c-c_0 \right\lVert \leqslant \delta \implies \left\lVert\gamma (c)-\gamma (c_0)\right\lVert=\left\lVert \varphi (c,c_0)-\varphi (c_0,c_0)\right\lVert \leqslant \epsilon \hspace{50px}\)

En particulier pour \(\epsilon = R(1-q)\), on a
:\(\left\lVert c-c_0 \right\lVert \leqslant \delta \implies \left\lVert\gamma (c)-\gamma (c_0)\right\lVert=\left\lVert \varphi (c,c_0)-\varphi (c_0,c_0)\right\lVert \leqslant R(1-q) \hspace{50px}\)

Posons \(r_0 = min(r,\delta)\),on a:

\(\left\lVert c-c_0 \right\lVert \leqslant r_0\),\(\left\lVert x-c_0 \right\lVert \leqslant R \implies \left\lVert\gamma (x)-\gamma (c_0)\right\lVert \leqslant \left\lVert \varphi (c,x)-\varphi(c,c_0) \right\lVert +\left\lVert \varphi (c,c_0)-\varphi(c_0,c_0) \right\lVert\)

\(\hspace{200px}\implies \left\lVert\gamma (c)-\gamma (c_0)\right\lVert\leqslant q\left\lVert x-c_0 \right\lVert +R(1-q) \hspace{80px}\)
d'après \((i)\) et le TAF

\(\hspace{200px}\implies \left\lVert\gamma (c)-\gamma (c_0)\right\lVert\leqslant qR+R(1-q)=R\)

Ainsi,
\(\left\lVert\gamma (c)-\gamma (c_0)\right\lVert _\infty= \sup{\{\left\lVert\varphi (c,x)-\varphi (c_0,c_0)\right\lVert, \left\lVert x-c_0 \right\lVert \leqslant r_0}\}\)

Et pour tout \(x_1,x_2\in B(c_0,R)\), on a:

\(\left\lVert\gamma (x_1)-\gamma (x_2) \right\lVert _\infty= \sup{\{\left\lVert\varphi (c,x_1)-\varphi (c,x_2)\right\lVert, \left\lVert x-c_0 \right\lVert \leqslant r_0}\}\)

\(\hspace{130px}\leqslant q\sup{ \{ \left\lVert x_1 -x_2 \right\lVert, \left\lVert x-c_0 \right\lVert \leqslant r_0 \} }\)
d'après (i) et le TAF

\(\hspace{130px}\leqslant q \left\lVert x_1 -x_2 \right\lVert _\infty\)

Donc la fonction \(\gamma\) est Lipschitzienne.

La fonction \(\gamma\) est continue et dérivable pour tout
\(c\in B(c_0,R)\).Ce qui nous permet de dire que la fonction \(\varphi\)
est \textbf{différentiable} pour tout \((c,x)\in B(c_0,r)\)x\(B(c_0,R)\)

\textbf{Etape 2:} montrons que \(\varphi\) est lipschitzienne.

On a
\(\varphi(c,x)=x-(\partial _x f(c_0,c_0))^{-1}(\frac{\partial f}{\partial x}(c,x))\)

Ainsi, pour tout \((c,x)\in B(c_0,r)\) x
\(B(c_0,R)\),\(\frac{\partial \varphi}{\partial x}(c,x)\) existe et on
a:

\(\frac{\partial \varphi}{\partial x}(c,x)=I_2-(\partial _x f(c_0,c_0))^{-1}(\frac{\partial f}{\partial x}(c,x))\)
où \(I_2: \mathbb{R}^2 \longrightarrow \mathbb{R}\) est l'application
identité.

Au point \(c_0\), on a:
\(\frac{\partial \varphi}{\partial x}(c_0,c_0)=I_2-(\partial _x f(c_0,c_0))^{-1}(\frac{\partial f}{\partial x}(c_0,c_0))=0\)

Donc \(\frac{\partial \varphi}{\partial x}(c_0,c_0)=0\) et (i) devient:

\(\forall x \in B \left\lVert c-c_0 \right\lVert \leqslant r,\left\lVert x-c_0 \right\lVert \leqslant R \implies \left\lVert \frac{\partial \varphi}{\partial x}(c,x)\right\lVert \leqslant q \hspace{50px}\)

Ainsi d'après le Théorème des Accroissements Finis, on a:

\(\forall x_1, x_2 \in B\),\(\left\lVert c-c_0 \right\lVert \leqslant r\),\$\left\lVert x-c\_0
\right\lVert \leqslant R
\implies \left\lVert \varphi (c,x\_1)-\varphi(c,x\_2)
\right\lVert \leqslant q \left\lVert x\_1-x\_2 \right\lVert \$

Donc la fonction \(\varphi\) est Lipschitzienne.

On conclut que la fonction \(\varphi (c,x)\) vérifie (3)

    \hypertarget{question-5preuve-dexistence-dun-prolongement-maximal-par-le-lemme-de-zorn}{%
\subsection{\texorpdfstring{\textbf{Question 5:Preuve d'existence d'un
prolongement maximal par le Lemme de
Zorn}}{Question 5:Preuve d'existence d'un prolongement maximal par le Lemme de Zorn}}\label{question-5preuve-dexistence-dun-prolongement-maximal-par-le-lemme-de-zorn}}

    Soit \(\gamma _0\) uner appliication telle que
\(\forall c\in I_0, f(c,\gamma(c_0))=0\).

Soit \(\gamma : I\longmapsto \mathbb{R}^N\) un prolongement de
\(\gamma _0\), on a:

\begin{quote}
\begin{enumerate}
\def\labelenumi{(\alph{enumi})}
\tightlist
\item
  \(I \supseteq I_0\) 
\item
  \(\forall c\in I_0, \gamma(c)=\gamma _0(c)\)
\end{enumerate}
\end{quote}

    Soit E l'ensemble des fonctions qui prolongent \(\gamma\).

Munissons cet ensemble de la relation ``\(\subseteq\)'' de sorte que
pour \(f,g \in E, f\subseteq g\) sinigfie qu'il existe un intervalle
dans lequel \(f=g\)

Alors \((E,\subseteq)\) ensemble inductif i.e pout tout \(i\neq j\) il
existe \(\gamma _i\) et \(\gamma _j\) dans E tels que
\(\gamma _i \subseteq \gamma _j\)

Le
\href{https://en.wikipedia.org/wiki/Zorn\%27s_lemma?oldformat=true}{lemme
de Zorn} nous assure que l'ensemble E ainsi définit admet un élément
maximal.

    \hypertarget{question-6}{%
\subsection{\texorpdfstring{\textbf{Question
6:}}{Question 6:}}\label{question-6}}

    Supposons qu'il existe deux extensions
\(\gamma _1:I_1 \longmapsto \mathbb{R}^N\) et
\(\gamma _2:I_2 \longmapsto \mathbb{R}^N\) de \(\gamma _0\); avec
\(I_1, I_2\subset[0,1]\)

Supposons que \(\gamma _1 \subseteq \gamma _2\)

On a pour tout \((c,x)\in \Omega f(c,\gamma _1(c))=0\) et
\(f(c,\gamma _2(c))=0\)

Par la règle de dérivation en chaine, on obtient:

\(\frac{\partial f}{\partial c}(c,\gamma _1(c))+\frac{\partial f}{\partial x}(c,\gamma _1(c)).\partial _c \gamma _1(c)=0\)

et
\(\frac{\partial f}{\partial c}(c,\gamma _2(c))+\frac{\partial f}{\partial x}(c,\gamma _2(c)).\partial _c \gamma _2(c)=0\)

Equivaut à:

\(\partial _c \gamma _1(c)=-(\frac{\partial f}{\partial x}(c,\gamma _1(c)))^{-1}(\frac{\partial f}{\partial c}(c,\gamma _1(c)))\)

et
\(\partial _c \gamma _2(c)=-(\frac{\partial f}{\partial x}(c,\gamma _2(c)))^{-1}(\frac{\partial f}{\partial c}(c,\gamma _2(c)))\)

(Preuve non terminée)

    \hypertarget{question-7}{%
\subsection{\texorpdfstring{\textbf{Question
7:}}{Question 7:}}\label{question-7}}

    Sous les hypothèses du Théorème des Fonctions Implicites,

En effet la fonction \(c\in I\longmapsto f(c,\gamma (c))\) étant
différentiable comme composée de fonctions différentiables, la
\href{https://fr.wikipedia.org/wiki/Th\%C3\%A9or\%C3\%A8me_de_d\%C3\%A9rivation_des_fonctions_compos\%C3\%A9es?oldformat=true}{règle
de dérivation par chaine}(chain rule) assure qu'en tout point de
\(\Omega\):

\(\frac{\partial f}{\partial c}(c,\gamma (c))+\frac{\partial f}{\partial x}(c,\gamma (c)).\partial _c \gamma (c)=0 \iff \partial _c \gamma (c)=(\frac{\partial f}{\partial x}(c,\gamma (c)))^{-1}.\frac{\partial f}{\partial c}(c,c)\)

    \hypertarget{question-8-routine-pour-la-ruxe9solution-de-ledo}{%
\subsection{\texorpdfstring{\textbf{Question 8:} Routine pour la
Résolution de
l'EDO}{Question 8: Routine pour la Résolution de l'EDO}}\label{question-8-routine-pour-la-ruxe9solution-de-ledo}}

    \begin{tcolorbox}[breakable, size=fbox, boxrule=1pt, pad at break*=1mm,colback=cellbackground, colframe=cellborder]
\prompt{In}{incolor}{13}{\boxspacing}
\begin{Verbatim}[commandchars=\\\{\}]
\PY{n}{f} \PY{o}{=} \PY{k}{lambda} \PY{n}{c}\PY{p}{,}\PY{n}{x}\PY{p}{:} \PY{n}{x}\PY{o}{*}\PY{o}{*}\PY{l+m+mi}{3}\PY{o}{+}\PY{n}{k\PYZus{}a}\PY{o}{*}\PY{n}{x}\PY{o}{*}\PY{o}{*}\PY{l+m+mi}{2}\PY{o}{\PYZhy{}}\PY{p}{(}\PY{n}{c}\PY{o}{*}\PY{n}{k\PYZus{}a}\PY{o}{+}\PY{n}{k\PYZus{}e}\PY{p}{)}\PY{o}{*}\PY{n}{x}\PY{o}{\PYZhy{}}\PY{n}{k\PYZus{}a}\PY{o}{*}\PY{n}{k\PYZus{}e}  \PY{c+c1}{\PYZsh{} Polynome P(c,x)}
\PY{n}{dfdc} \PY{o}{=} \PY{k}{lambda} \PY{n}{c}\PY{p}{,}\PY{n}{x}\PY{p}{:} \PY{o}{\PYZhy{}}\PY{n}{k\PYZus{}a}\PY{o}{*}\PY{n}{x}                            \PY{c+c1}{\PYZsh{} dérivée partielle de P par rapport à c}
\PY{n}{dfdx} \PY{o}{=} \PY{k}{lambda} \PY{n}{c}\PY{p}{,}\PY{n}{x}\PY{p}{:} \PY{l+m+mi}{3}\PY{o}{*}\PY{n}{x}\PY{o}{*}\PY{o}{*}\PY{l+m+mi}{2}\PY{o}{+}\PY{l+m+mi}{2}\PY{o}{*}\PY{n}{k\PYZus{}a}\PY{o}{*}\PY{n}{x}\PY{o}{\PYZhy{}}\PY{p}{(}\PY{n}{c}\PY{o}{*}\PY{n}{k\PYZus{}a}\PY{o}{+}\PY{n}{k\PYZus{}e}\PY{p}{)}        \PY{c+c1}{\PYZsh{} dérivée partielle de P par rapport à x}
\end{Verbatim}
\end{tcolorbox}

    \begin{tcolorbox}[breakable, size=fbox, boxrule=1pt, pad at break*=1mm,colback=cellbackground, colframe=cellborder]
\prompt{In}{incolor}{14}{\boxspacing}
\begin{Verbatim}[commandchars=\\\{\}]
\PY{k}{def} \PY{n+nf}{Routine}\PY{p}{(}\PY{n}{x0} \PY{o}{=} \PY{l+m+mi}{10}\PY{o}{*}\PY{o}{*}\PY{p}{(}\PY{o}{\PYZhy{}}\PY{l+m+mi}{8}\PY{p}{)}\PY{p}{)}\PY{p}{:}
    \PY{k}{def} \PY{n+nf}{gama}\PY{p}{(}\PY{n}{c}\PY{p}{,}\PY{n}{x}\PY{p}{)}\PY{p}{:}      \PY{c+c1}{\PYZsh{}définition de la fonction}
        \PY{k}{if} \PY{n}{dfdx}\PY{p}{(}\PY{n}{c}\PY{p}{,}\PY{n}{x}\PY{p}{)}\PY{o}{==}\PY{l+m+mi}{0}\PY{p}{:}    \PY{c+c1}{\PYZsh{} Traitement du cas ou dfdx n\PYZsq{}est pas inversible}
            \PY{k}{try}\PY{p}{:}
                \PY{n}{c} \PY{o}{=} \PY{n}{c} \PY{o}{\PYZhy{}} \PY{n+nb}{float}\PY{p}{(}\PY{n}{c}\PY{p}{)}\PY{o}{/}\PY{n}{dfdx}\PY{p}{(}\PY{n}{x}\PY{p}{)}
            \PY{k}{except} \PY{n+ne}{ZeroDivisionError}\PY{p}{:}
                \PY{n+nb}{print}\PY{p}{(}\PY{l+s+s2}{\PYZdq{}}\PY{l+s+s2}{Erreur! \PYZhy{} dérivée nulle pour c = }\PY{l+s+s2}{\PYZdq{}}\PY{p}{,} \PY{n}{c}\PY{p}{)}
        \PY{k}{return} \PY{o}{\PYZhy{}}\PY{n}{dfdc}\PY{p}{(}\PY{n}{c}\PY{p}{,}\PY{n}{x}\PY{p}{)}\PY{o}{/}\PY{n}{dfdx}\PY{p}{(}\PY{n}{c}\PY{p}{,}\PY{n}{x}\PY{p}{)}
    \PY{n}{points} \PY{o}{=} \PY{n}{np}\PY{o}{.}\PY{n}{linspace}\PY{p}{(}\PY{l+m+mi}{0}\PY{p}{,}\PY{l+m+mi}{1}\PY{p}{,}\PY{l+m+mi}{100}\PY{p}{)} \PY{c+c1}{\PYZsh{} points d\PYZsq{}évaluation}
    \PY{n}{sol} \PY{o}{=} \PY{n}{solve\PYZus{}ivp}\PY{p}{(}\PY{n}{fun}\PY{o}{=}\PY{n}{gama}\PY{p}{,}\PY{n}{t\PYZus{}span}\PY{o}{=}\PY{p}{[}\PY{l+m+mi}{0}\PY{p}{,} \PY{l+m+mi}{1}\PY{p}{]}\PY{p}{,} \PY{n}{y0}\PY{o}{=}\PY{p}{[}\PY{n}{x0}\PY{p}{]}\PY{p}{,}\PY{n}{t\PYZus{}eval}\PY{o}{=}\PY{n}{points}\PY{p}{)}
    \PY{k}{return} \PY{n}{sol}\PY{o}{.}\PY{n}{y}
\end{Verbatim}
\end{tcolorbox}

    \begin{tcolorbox}[breakable, size=fbox, boxrule=1pt, pad at break*=1mm,colback=cellbackground, colframe=cellborder]
\prompt{In}{incolor}{15}{\boxspacing}
\begin{Verbatim}[commandchars=\\\{\}]
\PY{n}{Routine}\PY{p}{(}\PY{p}{)}\PY{p}{;}
\end{Verbatim}
\end{tcolorbox}

    \begin{tcolorbox}[breakable, size=fbox, boxrule=1pt, pad at break*=1mm,colback=cellbackground, colframe=cellborder]
\prompt{In}{incolor}{16}{\boxspacing}
\begin{Verbatim}[commandchars=\\\{\}]
\PY{n}{c} \PY{o}{=} \PY{n}{np}\PY{o}{.}\PY{n}{linspace}\PY{p}{(}\PY{l+m+mi}{0}\PY{p}{,}\PY{l+m+mi}{1}\PY{p}{,}\PY{l+m+mi}{100}\PY{p}{)} \PY{c+c1}{\PYZsh{} points d\PYZsq{}évaluation}
\PY{n}{gama} \PY{o}{=} \PY{n}{Routine}\PY{p}{(}\PY{p}{)}\PY{o}{.}\PY{n}{T}        \PY{c+c1}{\PYZsh{} transposée}
\PY{n}{plt}\PY{o}{.}\PY{n}{plot}\PY{p}{(}\PY{n}{c}\PY{p}{,} \PY{n}{gama}\PY{p}{)}
\PY{n}{plt}\PY{o}{.}\PY{n}{xlabel}\PY{p}{(}\PY{l+s+s1}{\PYZsq{}}\PY{l+s+s1}{c}\PY{l+s+s1}{\PYZsq{}}\PY{p}{)}
\PY{n}{plt}\PY{o}{.}\PY{n}{title}\PY{p}{(}\PY{l+s+s1}{\PYZsq{}}\PY{l+s+s1}{Courbre des solutions}\PY{l+s+s1}{\PYZsq{}}\PY{p}{)}
\PY{n}{plt}\PY{o}{.}\PY{n}{show}\PY{p}{(}\PY{p}{)}
\end{Verbatim}
\end{tcolorbox}

    \begin{center}
    \adjustimage{max size={0.9\linewidth}{0.9\paperheight}}{output_48_0.png}
    \end{center}
    { \hspace*{\fill} \\}
    
    \hypertarget{question-9-influence-des-paramuxe8tres-rtol-et-atol-de-scipy.integratesolve_ivp}{%
\subsection{\texorpdfstring{\textbf{Question 9:} Influence des
paramètres \texttt{rtol} et \texttt{atol} de
\texttt{scipy.integrate:solve\_ivp}}{Question 9: Influence des paramètres rtol et atol de scipy.integrate:solve\_ivp}}\label{question-9-influence-des-paramuxe8tres-rtol-et-atol-de-scipy.integratesolve_ivp}}

    \begin{tcolorbox}[breakable, size=fbox, boxrule=1pt, pad at break*=1mm,colback=cellbackground, colframe=cellborder]
\prompt{In}{incolor}{18}{\boxspacing}
\begin{Verbatim}[commandchars=\\\{\}]
\PY{n}{f} \PY{o}{=} \PY{k}{lambda} \PY{n}{c}\PY{p}{,}\PY{n}{x}\PY{p}{:} \PY{n}{x}\PY{o}{*}\PY{o}{*}\PY{l+m+mi}{3}\PY{o}{+}\PY{n}{k\PYZus{}a}\PY{o}{*}\PY{n}{x}\PY{o}{*}\PY{o}{*}\PY{l+m+mi}{2}\PY{o}{\PYZhy{}}\PY{p}{(}\PY{n}{c}\PY{o}{*}\PY{n}{k\PYZus{}a}\PY{o}{+}\PY{n}{k\PYZus{}e}\PY{p}{)}\PY{o}{*}\PY{n}{x}\PY{o}{\PYZhy{}}\PY{n}{k\PYZus{}a}\PY{o}{*}\PY{n}{k\PYZus{}e}
\PY{n}{c} \PY{o}{=} \PY{n}{np}\PY{o}{.}\PY{n}{linspace}\PY{p}{(}\PY{l+m+mi}{0}\PY{p}{,}\PY{l+m+mi}{1}\PY{p}{,}\PY{l+m+mi}{100}\PY{p}{)}
\PY{n}{x}\PY{o}{=}\PY{n}{f}\PY{p}{(}\PY{n}{c}\PY{p}{,}\PY{n}{gama}\PY{p}{)}
\PY{n}{x1} \PY{o}{=} \PY{n}{f}\PY{p}{(}\PY{l+m+mi}{1}\PY{p}{,}\PY{n}{gama}\PY{p}{)}
\PY{n}{norm\PYZus{}f} \PY{o}{=} \PY{n}{np}\PY{o}{.}\PY{n}{linalg}\PY{o}{.}\PY{n}{norm}\PY{p}{(}\PY{n}{x}\PY{p}{,}\PY{n+nb}{ord}\PY{o}{=}\PY{l+m+mi}{2}\PY{p}{)}
\PY{n}{norm\PYZus{}f\PYZus{}c1} \PY{o}{=} \PY{n}{np}\PY{o}{.}\PY{n}{linalg}\PY{o}{.}\PY{n}{norm}\PY{p}{(}\PY{n}{x1}\PY{p}{,}\PY{n+nb}{ord}\PY{o}{=}\PY{l+m+mi}{2}\PY{p}{)}
\PY{n}{norm\PYZus{}f}\PY{p}{,} \PY{n}{norm\PYZus{}f\PYZus{}c1}
\end{Verbatim}
\end{tcolorbox}

            \begin{tcolorbox}[breakable, size=fbox, boxrule=.5pt, pad at break*=1mm, opacityfill=0]
\prompt{Out}{outcolor}{18}{\boxspacing}
\begin{Verbatim}[commandchars=\\\{\}]
(1.9735368185305545e-06, 2.1857547068173576e-07)
\end{Verbatim}
\end{tcolorbox}
        
    \hypertarget{question-10}{%
\subsection{\texorpdfstring{\textbf{Question
10:}}{Question 10:}}\label{question-10}}

    \begin{tcolorbox}[breakable, size=fbox, boxrule=1pt, pad at break*=1mm,colback=cellbackground, colframe=cellborder]
\prompt{In}{incolor}{57}{\boxspacing}
\begin{Verbatim}[commandchars=\\\{\}]
\PY{c+c1}{\PYZsh{} Fonction pour determiner la matrice Jacobienne de h}
\PY{k}{def} \PY{n+nf}{J}\PY{p}{(}\PY{n}{h}\PY{p}{,} \PY{n}{x}\PY{p}{,} \PY{n}{dx}\PY{o}{=}\PY{l+m+mf}{1e\PYZhy{}8}\PY{p}{)}\PY{p}{:}
    \PY{n}{n} \PY{o}{=} \PY{n+nb}{len}\PY{p}{(}\PY{n}{x}\PY{p}{)}
    \PY{n}{func} \PY{o}{=} \PY{n}{h}\PY{p}{(}\PY{p}{(}\PY{n}{x}\PY{p}{)}\PY{p}{)}
    \PY{n}{jac} \PY{o}{=} \PY{n}{np}\PY{o}{.}\PY{n}{zeros}\PY{p}{(}\PY{p}{(}\PY{n}{n}\PY{p}{,} \PY{n}{n}\PY{p}{)}\PY{p}{)}
    \PY{k}{for} \PY{n}{j} \PY{o+ow}{in} \PY{n+nb}{range}\PY{p}{(}\PY{n}{n}\PY{p}{)}\PY{p}{:}  \PY{c+c1}{\PYZsh{} compteur sur les colonnes}
        \PY{n}{Dxj} \PY{o}{=} \PY{p}{(}\PY{n+nb}{abs}\PY{p}{(}\PY{n}{x}\PY{p}{[}\PY{n}{j}\PY{p}{]}\PY{p}{)}\PY{o}{*}\PY{n}{dx} \PY{k}{if} \PY{n}{x}\PY{p}{[}\PY{n}{j}\PY{p}{]} \PY{o}{!=} \PY{l+m+mi}{0} \PY{k}{else} \PY{n}{dx}\PY{p}{)}
        \PY{n}{x\PYZus{}plus} \PY{o}{=} \PY{p}{[}\PY{p}{(}\PY{n}{xi} \PY{k}{if} \PY{n}{k} \PY{o}{!=} \PY{n}{j} \PY{k}{else} \PY{n}{xi} \PY{o}{+} \PY{n}{Dxj}\PY{p}{)} \PY{k}{for} \PY{n}{k}\PY{p}{,} \PY{n}{xi} \PY{o+ow}{in} \PY{n+nb}{enumerate}\PY{p}{(}\PY{n}{x}\PY{p}{)}\PY{p}{]}
        \PY{n}{jac}\PY{p}{[}\PY{p}{:}\PY{p}{,} \PY{n}{j}\PY{p}{]} \PY{o}{=} \PY{p}{(}\PY{n}{h}\PY{p}{(}\PY{n}{x\PYZus{}plus}\PY{p}{)} \PY{o}{\PYZhy{}} \PY{n}{func}\PY{p}{)}\PY{o}{/}\PY{n}{Dxj}
    \PY{k}{return} \PY{n}{jac}
\end{Verbatim}
\end{tcolorbox}

    \begin{tcolorbox}[breakable, size=fbox, boxrule=1pt, pad at break*=1mm,colback=cellbackground, colframe=cellborder]
\prompt{In}{incolor}{ }{\boxspacing}
\begin{Verbatim}[commandchars=\\\{\}]
\PY{k}{def} \PY{n+nf}{newton}\PY{p}{(}\PY{n}{h}\PY{p}{,}\PY{n}{x0}\PY{p}{,} \PY{n}{tol}\PY{p}{,} \PY{n}{max\PYZus{}iter}\PY{p}{)}\PY{p}{:}
    \PY{k}{for} \PY{n}{n} \PY{o+ow}{in} \PY{n+nb}{range}\PY{p}{(}\PY{l+m+mi}{0}\PY{p}{,}\PY{n}{max\PYZus{}iter}\PY{p}{)}\PY{p}{:}
        \PY{n}{hx0} \PY{o}{=} \PY{n}{h}\PY{p}{(}\PY{n}{x0}\PY{p}{)}
        \PY{k}{if} \PY{n+nb}{abs}\PY{p}{(}\PY{n}{hx0}\PY{p}{)} \PY{o}{\PYZlt{}} \PY{n}{tol}\PY{p}{:}
            \PY{n+nb}{print}\PY{p}{(}\PY{l+s+s2}{\PYZdq{}}\PY{l+s+s2}{solution trouvée après }\PY{l+s+si}{\PYZob{}\PYZcb{}}\PY{l+s+s2}{ itérations.}\PY{l+s+s2}{\PYZdq{}}\PY{o}{.}\PY{n}{format}\PY{p}{(}\PY{n}{n}\PY{p}{)}\PY{p}{)}
            \PY{k}{return} \PY{n}{x0}
        \PY{n}{Jhxn} \PY{o}{=} \PY{c+c1}{\PYZsh{} A revoir}
        \PY{k}{if} \PY{n}{Jhxn} \PY{o}{==} \PY{l+m+mi}{0}\PY{p}{:}
            \PY{n+nb}{print}\PY{p}{(}\PY{l+s+s2}{\PYZdq{}}\PY{l+s+s2}{Jacobienne Singulière. Pas de solution}\PY{l+s+s2}{\PYZdq{}}\PY{p}{)}
            \PY{k}{return} \PY{k+kc}{None}
        \PY{n}{xn} \PY{o}{=} \PY{n}{xn} \PY{o}{\PYZhy{}} \PY{n}{hxn}\PY{o}{/}\PY{n}{Jhxn}
    \PY{n+nb}{print}\PY{p}{(}\PY{l+s+s2}{\PYZdq{}}\PY{l+s+s2}{Aucune solution trouvée après }\PY{l+s+si}{\PYZob{}\PYZcb{}}\PY{l+s+s2}{ itérations}\PY{l+s+s2}{\PYZdq{}}\PY{o}{.}\PY{n}{format}\PY{p}{(}\PY{n}{max\PYZus{}iter}\PY{p}{)}\PY{p}{)}
    \PY{k}{return} \PY{k+kc}{None}
\end{Verbatim}
\end{tcolorbox}

    \hypertarget{question-11}{%
\subsection{\texorpdfstring{\textbf{Question
11:}}{Question 11:}}\label{question-11}}

    \begin{tcolorbox}[breakable, size=fbox, boxrule=1pt, pad at break*=1mm,colback=cellbackground, colframe=cellborder]
\prompt{In}{incolor}{ }{\boxspacing}
\begin{Verbatim}[commandchars=\\\{\}]
\PY{c+c1}{\PYZsh{} non terminee}
\PY{k}{def} \PY{n+nf}{Newton}\PY{p}{(}\PY{n}{h}\PY{p}{,}\PY{p}{(}\PY{n}{x0}\PY{p}{)}\PY{p}{,} \PY{n}{tol}\PY{p}{,} \PY{n}{max\PYZus{}iter}\PY{p}{)}\PY{p}{:}
    \PY{n}{Jh} \PY{o}{=} \PY{c+c1}{\PYZsh{} matrice Jacobienne}
    \PY{n}{Jhx0} \PY{o}{=} \PY{n}{Jh}\PY{p}{(}\PY{n}{x0}\PY{p}{)}
    \PY{n}{xtild} \PY{o}{=} \PY{n}{Routine}\PY{p}{(}\PY{n}{x0}\PY{p}{)}   \PY{c+c1}{\PYZsh{} approximation à l\PYZsq{}ordre 0}
    \PY{k}{for} \PY{n}{n} \PY{o+ow}{in} \PY{n+nb}{range}\PY{p}{(}\PY{l+m+mi}{0}\PY{p}{,}\PY{n}{max\PYZus{}iter}\PY{p}{)}\PY{p}{:}
        \PY{n}{x} \PY{o}{=} \PY{n}{xtild} \PY{o}{+} \PY{n}{hx0}\PY{o}{/}\PY{n}{Jh}\PY{p}{(}\PY{n}{xtild}\PY{p}{)}
        \PY{n}{Jhxn} \PY{o}{=} \PY{n}{Jh}\PY{p}{(}\PY{n}{x}\PY{p}{)}\PY{c+c1}{\PYZsh{} A revoir}
        \PY{n}{hx} \PY{o}{=} \PY{n}{h}\PY{p}{(}\PY{n}{x}\PY{p}{)}
        \PY{n}{norm\PYZus{}h} \PY{o}{=} \PY{n}{np}\PY{o}{.}\PY{n}{linalg}\PY{o}{.}\PY{n}{norm}\PY{p}{(}\PY{n}{hx}\PY{p}{,} \PY{n+nb}{ord}\PY{o}{=}\PY{l+m+mi}{2}\PY{p}{)}
        \PY{k}{if} \PY{n+nb}{abs}\PY{p}{(}\PY{n}{norm\PYZus{}h}\PY{p}{)} \PY{o}{\PYZgt{}} \PY{n}{tol}\PY{p}{:}
            \PY{k}{continue}
    \PY{k}{return} \PY{n}{x}
\end{Verbatim}
\end{tcolorbox}

    \hypertarget{conclusion}{%
\subsubsection{\texorpdfstring{\textbf{Conclusion:}}{Conclusion:}}\label{conclusion}}

Il convient de rappeller que ce travail est proposé en vue d'une
évaluation par mes encadreurs. Il est donc possible qu'il y ai quelques
coquilles. Merci pour votre comprehension.

    \hypertarget{refuxe9rences}{%
\subsubsection{\texorpdfstring{\textbf{Reférences:}}{Reférences:}}\label{refuxe9rences}}

\begin{enumerate}
\def\labelenumi{\arabic{enumi}.}
\tightlist
\item
  Cours \href{}{Introduction to Numerical Analysis and Statistical
  Modeling} BAB2 Umons par Christophe Troestler\\
\item
  \href{https://pythonnumericalmethods.berkeley.edu/notebooks/Index.html}{Python
  Programming and Numerical Methods - A Guide for Engineers and
  Scientists} by Qingkai Kong, Timmy Siauw, Alexandre Bayen (Authors)
\item
  Earl A. Coddington, Norman Levinson - \href{}{Theory of ordinary
  differential equations-R.E. Krieger} (1984)
\item
  Wikipedia
  \href{https://en.wikipedia.org/wiki/Numerical_differentiation?oldformat=true}{Numerical
  differentiation} et
  \href{https://en.wikipedia.org/w/index.php?title=Numerical_methods_for_ordinary_differential_equations\&oldformat=true}{Numerical
  methods for ordinary differential equations}
\end{enumerate}

    \begin{tcolorbox}[breakable, size=fbox, boxrule=1pt, pad at break*=1mm,colback=cellbackground, colframe=cellborder]
\prompt{In}{incolor}{ }{\boxspacing}
\begin{Verbatim}[commandchars=\\\{\}]

\end{Verbatim}
\end{tcolorbox}


    % Add a bibliography block to the postdoc
    
    
    
\end{document}
